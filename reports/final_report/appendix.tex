

\section{Detailed Rally scenarios}
\label{sec:detail-rally}

(Extract from the blog article, from the work of Ronan Cherrueau)

The \%reads/\%writes ratio is computed on Mariadb. The gauging code reads values of status variables \verb+Com\_xxx+ that provide statement counts over all connections (with \verb+xxx+ standing for \verb+SELECT+, \verb+DELETE+, \verb+INSERT+, \verb+UPDATE+, \verb+REPLACE+ statements). The SQL query that does this job is available in listing \ref{lst:rw} and returns the total number of reads and writes since the database started. That SQL query is called before and after the execution of one Rally scenario. After and before values are then subtracted to compute the number of reads and writes performed during the scenario and finally, compared to compute the ratio.

\begin{lstlisting}[caption={Total number of reads and writes performed on MariaDB since the last reboot}\label{lst:rw}, language=SQL, captionpos=b]
SELECT
  SUM(IF(variable_name = 'Com_select', variable_value, 0))
     AS `Total reads`,
  SUM(IF(variable_name IN ('Com_delete',
                           'Com_insert',
                           'Com_update',
                           'Com_replace'), variable_value, 0))
     AS `Total writes`
FROM  information_schema.GLOBAL_STATUS;
\end{lstlisting}

Note that \%reads/\%writes may be a little bit more in favor of reads than what it is presented here because the following also takes into account the creation/deletion of the Rally context. A basic Rally context for a Keystone scenario is \verb+{"admin_cleanup@openstack": ["keystone"]}+. We are not sure what does this context do exactly though, maybe it only creates an admin user… This context may be extended by other inserts specified in the scenario definition (under the \verb+context+ key; see scenario definition for keystone/create-add-and-list-user-roles).

The Juice implementation for this gauging is available on GitHub at \url{https://github.com/rcherrueau/juice/blob/02af922a7c3221462d7106dfb2751b3be709a4d5/experiments/read-write-ratio.py}.


\subsection{keystone/authenticate-user-and-validate-token}

Description: authenticate and validate a keystone token.

Definition Code: \url{https://github.com/openstack/rally-openstack/blob/master/samples/tasks/scenarios/keystone/authenticate-user-and-validate-token.yaml}

Source Code: \url{https://github.com/openstack/rally-openstack/blob/master/rally\_openstack/scenarios/keystone/basic.py#L111-L120}

List of keystone functionalities:
\begin{itemize}
\item    keystone\_v3.fetch\_token
\item    keystone\_v3.validate\_token
\end{itemize}
\%Reads/\%Writes: 96.46/3.54

Number of runs: 20

\subsection{keystone/create-add-and-list-user-roles}

Description: create user role, add it and list user roles for given user.

Definition Code: \url{https://github.com/openstack/rally-openstack/blob/master/samples/tasks/scenarios/keystone/create-add-and-list-user-roles.yaml}

Source Code: \url{https://github.com/openstack/rally-openstack/blob/master/rally\_openstack/scenarios/keystone/basic.py#L214-L228}

List of keystone functionalities:
\begin{itemize}
\item    keystone\_v3.create\_role
\item    keystone\_v3.add\_role
\item    keystone\_v3.list\_roles
\end{itemize}
\%Reads/\%Writes: 96.22/3.78

Number of runs: 100

\subsection{keystone/create-and-list-tenants}

Description: create a keystone tenant with random name and list all tenants.

Definition Code: \url{https://github.com/openstack/rally-openstack/blob/master/samples/tasks/scenarios/keystone/create-and-list-tenants.yaml}

Source Code: \url{https://github.com/openstack/rally-openstack/blob/master/rally\_openstack/scenarios/keystone/basic.py#L166-L181}

List of keystone functionalities
\begin{itemize}
\item    keystone\_v3.create\_project
\item    keystone\_v3.list\_projects
\end{itemize}
\%Reads/\%Writes: 92.12/7.88

Number of runs: 10

\subsection{keystone/get-entities}

Description: get instance of a tenant, user, role and service by id’s. An ephemeral tenant, user, and role are each created. By default, fetches the ’keystone’ service.

List of keystone functionalities:
\begin{itemize}
\item    keystone\_v3.create\_project
\item    keystone\_v3.create\_user
\item    keystone\_v3.create\_role
    \begin{itemize}
    \item    keystone\_v3.list\_roles
    \item    keystone\_v3.add\_role
    \end{itemize}
\item    keystone\_v3.get\_project
\item    keystone\_v3.get\_user
\item    keystone\_v3.get\_role
\item    keystone\_v3.list\_services
\item    keystone\_v3.get\_services
\end{itemize}
\%Reads/\%Writes: 91.9/8.1

Definition Code: \url{https://github.com/openstack/rally-openstack/blob/master/samples/tasks/scenarios/keystone/get-entities.yaml}

Source Code: \url{https://github.com/openstack/rally-openstack/blob/master/rally\_openstack/scenarios/keystone/basic.py#L231-L261}

Number of runs: 100


\subsection{keystone/create-user-update-password}

Description: create user and update password for that user.

List of keystone functionalities:
\begin{itemize}
\item    keystone\_v3.create\_user
\item    keystone\_v3.update\_user
\end{itemize}
\%Reads/\%Writes: 89.79/10.21

Definition Code: \url{https://github.com/openstack/rally-openstack/blob/master/samples/tasks/scenarios/keystone/create-user-update-password.yaml}

Source Code: \url{https://github.com/openstack/rally-openstack/blob/master/rally\_openstack/scenarios/keystone/basic.py#L306-L320}

Number of runs: 100


\subsection{keystone/create-user-set-enabled-and-delete}

Description: create a keystone user, enable or disable it, and delete it.

List of keystone functionalities:
\begin{itemize}
\item    keystone\_v3.create\_user
\item    keystone\_v3.update\_user
\item    keystone\_v3.delete\_user
\end{itemize}
\%Reads/\%Writes: 91.07/8.93

Definition Code: \url{https://github.com/openstack/rally-openstack/blob/master/samples/tasks/scenarios/keystone/create-user-set-enabled-and-delete.yaml}

Source Code: \url{https://github.com/openstack/rally-openstack/blob/master/rally\_openstack/scenarios/keystone/basic.py#L75-L91}

Number of runs: 100

\subsection{keystone/create-and-list-users}

Description: create a keystone user with random name and list all users.

List of keystone functionalities:
\begin{itemize}
\item    keystone\_v3.create\_user
\item    keystone\_v3.list\_users
\end{itemize}
\%Reads/\%Writes: 92.05/7.95

Definition Code: \url{https://github.com/openstack/rally-openstack/blob/master/samples/tasks/scenarios/keystone/create-and-list-users.yaml}

Source Code: \url{https://github.com/openstack/rally-openstack/blob/master/rally\_openstack/scenarios/keystone/basic.py#L145-L163}

Number of runs: 100

\section{Results}
\label{results}

\subsection{Number of OpenStack instances impact under light load}

Figures \ref{fig:oss-impact-light}, \ref{fig:oss-impact-light-cdf} and \ref{fig:oss-impact-light-lreg} result from the experiments about the number of OpenStack instances impact under a light load.


\begin{figure}[h]
  %\vspace{-10pt}
  \centering
  \centerline{\includegraphics[width=1.4\textwidth, angle=90]{no_push/oss-impact-light.png}}
  %\vspace{-5pt}
  \caption{Impact of the Number of OpenStack Instances on the Completion Time under Light Load – Median Time for Each Operations.}
  %\vspace{-5pt}
  \label{fig:oss-impact-light}
\end{figure}


\begin{figure}[h]
%  \vspace{-10pt}
  \centering
  \centerline{\includegraphics[width=1.4\textwidth, angle=90]{no_push/oss-impact-light-cdf.png}}
%  \vspace{-5pt}
  \caption{Impact of the Number of OpenStack Instances on the Completion Time under Light Load – Cummulative Distibution}
%  \vspace{-5pt}
  \label{fig:oss-impact-light-cdf}
\end{figure}


\begin{figure}[h]
%  \vspace{-10pt}
  \centering
  \centerline{\includegraphics[width=1.4\textwidth, angle=90]{no_push/oss-impact-light-lreg.png}}
%  \vspace{-5pt}
  \caption{Impact of the Number of OpenStack Instances on the Completion Time under Light Load – Linear Regression}
%  \vspace{-5pt}
  \label{fig:oss-impact-light-lreg}
\end{figure}

\subsection{Number of OpenStack instances impact under heavy load}

Figures \ref{fig:oss-impact-high} and \ref{fig:oss-impact-high-cdf} result from the experiments about the number of OpenStack instances impact under a heavy load.


\begin{figure}[h]
%  \vspace{-10pt}
  \centering
  \centerline{\includegraphics[width=1.4\textwidth, angle=90]{no_push/oss-impact-high.png}}
%  \vspace{-5pt}
  \caption{Impact of the Number of OpenStack Instances on the Completion Time under Heavy Load – Median Time for Each Operations.}
%  \vspace{-5pt}
  \label{fig:oss-impact-high}
\end{figure}


\begin{figure}[h]
%  \vspace{-10pt}
  \centering
  \centerline{\includegraphics[width=1.4\textwidth, angle=90]{no_push/oss-impact-high-cdf.png}}
%  \vspace{-5pt}
  \caption{Impact of the Number of OpenStack Instances on the Completion Time under Heavy Load – Cummulative Distibution}
%  \vspace{-5pt}
  \label{fig:oss-impact-high-cdf}
\end{figure}

\subsection{Delay impact under light load}

Figures \ref{fig:delay-impact-light} and \ref{fig:delay-impact-light-cdf} result from the experiments about the delay impact under a light load.

\begin{figure}[h]
%  \vspace{-10pt}
  \centering
  \centerline{\includegraphics[width=1.4\textwidth, angle=90]{no_push/delay-impact-light.png}}
%  \vspace{-5pt}
  \caption{Impact of the Delay on the Completion Time under Light Load – Median Time for Each Operations.}
%  \vspace{-5pt}
  \label{fig:delay-impact-light}
\end{figure}

\begin{figure}[h]
%  \vspace{-10pt}
  \centering
  \centerline{\includegraphics[width=1.4\textwidth, angle=90]{no_push/delay-impact-light-cdf.png}}
%  \vspace{-5pt}
  \caption{Impact of the Delay on the Completion Time under Light Load – Cummulative Distibution.}
%  \vspace{-5pt}
  \label{fig:delay-impact-light-cdf}
\end{figure}


\subsection{Delay impact under heavy load}
Figures \ref{fig:delay-impact-high} and \ref{fig:delay-impact-high-cdf} result from the experiments about the delay impact under a heavy load.


\begin{figure}[h]
%  \vspace{-10pt}
  \centering
  \centerline{\includegraphics[width=1.4\textwidth, angle=90]{no_push/delay-impact-high.png}}
%  \vspace{-5pt}
  \caption{Impact of the Delay on the Completion Time under Heavy Load – Median Time for Each Operations.}
%  \vspace{-5pt}
  \label{fig:delay-impact-high}
\end{figure}

\begin{figure}[h]
%  \vspace{-10pt}
  \centering
  \centerline{\includegraphics[width=1.4\textwidth, angle=90]{no_push/delay-impact-high-cdf.png}}
%  \vspace{-5pt}
  \caption{Impact of the Delay on the Completion Time under Heavy Load – Cummulative Distibution.}
%  \vspace{-5pt}
  \label{fig:delay-impact-high-cdf}
\end{figure}

\subsection{Locality impact under light load}

Figures \ref{fig:zones-impact-light} and \ref{fig:zones-impact-light-cdf} result from the experiments about the locality impact under a light load.

\begin{figure}[h]
%  \vspace{-10pt}
  \centering
  \centerline{\includegraphics[width=1.4\textwidth, angle=90]{no_push/zones-impact-light.png}}
%  \vspace{-5pt}
  \caption{Impact of the Locality on the Completion Time under Light Load – Median Time for Each Operations.}
%  \vspace{-5pt}
  \label{fig:zones-impact-light}
\end{figure}

\begin{figure}[h]
%  \vspace{-10pt}
  \centering
  \centerline{\includegraphics[width=1.4\textwidth, angle=90]{no_push/zones-impact-light-cdf.png}}
%  \vspace{-5pt}
  \caption{Impact of the Locality on the Completion Time under Light Load – Cummulative Distibution.}
%  \vspace{-5pt}
  \label{fig:zones-impact-light-cdf}
\end{figure}
