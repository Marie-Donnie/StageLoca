An ever growing number of applications requires a wide distribution over the globe of computing resourcing, such as the Internet of Things (IoT). To manage this kind of ``Edge Computing'', the Discovery initiative chose to modify the OpenStack platform in order to avoid reinventing the wheel.

Some OpenStack components do not fulfill edge infrastructure needs. Especially, the Relational DataBase Management System is difficult to distribute accross the infrastructure because it has to cope with lots of connections and high latency network, making it crucial to consider the locality of the client.

Keystone, OpenStack authentication service, is using by default MariaDB to store all its data. Moreover, experimentations made on Keystone and their results make sense for other OpenStack services (Nova, Neutron, \dots) which also uses MariaDB by default.

To test the performances of different backends for Keystone and to better have an idea of the behavior of OpenStack in the context of the edge using locality, we made an evaluation of the performances of different RDBMS, using different topologies.

In order to get this evaluation, I developed Juice, a framework to deploy OpenStack using different backends on a varying number of servers, varying network delays, and a simulation of an edge infrastructure, with heterogenous network delays.

The results of this evaluation show that, though it is a single point of failure, a centralized MariaDB works pretty well under an increasing number of nodes and latency. MariaDB Galera Cluster handles pretty well high latencies despite what we first thought, but has serious difficulties to cope with a high number of nodes. Finally, CockroachdB manages pretty well numerous nodes, but can't handle high delays.

What was even more interesting were the results from the usage of the replication zones from CockroachDB. This showed that we can bypass the difficulties of a distributed database by considering the locality of the client. There are still disadvantages about CockroachDB; e.g. the fact that the locality is parametrized generally (for all the database), and we want to be able to configure it per query, e.g. add a user where he will use services (for Keystone), get the state of a booted VM on a server (for Nova). Parametrization per query ensures that data are written where they are supposed to be, closer to the client, which provides a resiliency to disconnections and network partitions.

This means that, though CockroachDB is not entirely what we need, it has proved that considering the locality of the client is a good way to continue the work for the Discovery initiative. This parametrization per query raises new challenges such as where to put the locality information and modify the queries to consider it.
