\section{Number of OpenStack Instances Impact}

This test evaluates how the completion time of Rally Keystone’s scenarios varies, depending on the RDBMS and the number of OpenStack instances. It measures the capacity of a RDBMS to handle lots of connections and requests. In this test, the number of OpenStack instances varies between 3, 9 and 45 and a LAN link inter-connects instances. As explained before, the deployment of the database depends on the RDBMS. With MariaDB, one instance of OpenStack contains the database, and others connect to that one. For Galera and CockroachDB, each OpenStack contains an instance of the RDBMS.

For these experiments, Juice deployed the RDBMS along with OpenStack instances and plays Rally scenarios listed in section \ref{rally-scenarios} page \pageref{rally-scenarios}. Juice runs Rally scenarios in both single and heavy modes. Results are presented in two next sections. The Juice implementation for these experiments is available on GitHub at \url{https://github.com/BeyondTheClouds/juice/tree/master/experiments}, except for the locality, which is available at \url{https://github.com/Marie-Donnie/juice/blob/regions/experiments.py}. This experiments defines the different groups and the delays to apply on the network, inside and between the groups.

We then filter experiments to only keep those when delay is 0 in a single Rally mode. Groups results by scenario’s name, RDBMS technology and number of OpenStack instances. Then, filters results above the 95th quantile. In the plot, the $\lambda$ Greek letter stands for the failure rate and $\sigma$ for the standard deviation.

\subsection{Light load}

Figure \ref{fig:oss-impact-light} presents the mean completion time (in seconds) of Keystone scenarios in a light Rally mode. In the figure, columns presents results of a specific scenario: the first column presents results for Authenticate User and Validate Token, the second for Create Add and List User Role, and so forth. Rows present results with a specific RDBMS: the first row presents results for MariaDB, second for Galera and third for CockroachDB. The figure presents results with stacked bar charts. Each bar presents the result for a specific number of OpenStack instances (i.e., 3, 9 and 45) and stacks completion times of each Keystone operations inside a scenario.

Cdf\ref{fig:oss-impact-light-cdf}


\begin{figure}[H]
  \vspace{-10pt}
  \centering
  \centerline{\includegraphics[width=1\textwidth]{no_push/short-oss-impact-light.png}}
  \vspace{-5pt}
  \caption{Impact of the Number of OpenStack Instances on the Completion Time under Light Load – Median Time for Each Operations.}
  \vspace{-5pt}
  \label{fig:short-oss-impact-light}
\end{figure}

\subsection{Heavy load}

\begin{figure}[H]
  \vspace{-10pt}
  \centering
  \centerline{\includegraphics[width=1\textwidth]{no_push/short-oss-impact-high.png}}
  \vspace{-5pt}
  \caption{Impact of the Number of OpenStack Instances on the Completion Time under Heavy Load – Median Time for Each Operations.}
  \vspace{-5pt}
  \label{fig:short-oss-impact-high}
\end{figure}


\section{Delay Impact}

In this test, the size of the database cluster is 9 and the delay varies between LAN, 100 and 300 ms of RTT. The test evaluates how the completion time of Rally scenarios varies, depending of RTT between nodes of the swarm.



\subsection{Light load}

\begin{figure}[H]
  \vspace{-10pt}
  \centering
  \centerline{\includegraphics[width=1\textwidth]{no_push/short-delay-impact-light.png}}
  \vspace{-5pt}
  \caption{Impact of the Delay on the Completion Time under Light Load – Median Time for Each Operations.}
  \vspace{-5pt}
  \label{fig:short-delay-impact-light}
\end{figure}

Figure \ref{fig:delay-impact-light}

cdf\ref{fig:delay-impact-light-cdf}

\subsection{Heavy load}

\begin{figure}[H]
  \vspace{-10pt}
  \centering
  \centerline{\includegraphics[width=1\textwidth]{no_push/short-delay-impact-high.png}}
  \vspace{-5pt}
  \caption{Impact of the Delay on the Completion Time under Heavy Load – Median Time for Each Operations.}
  \vspace{-5pt}
  \label{fig:short-delay-impact-high}
\end{figure}

Figure \ref{fig:delay-impact-high}

cdf \ref{fig:delay-impact-high-cdf}

\section{Locality Impact}

This study considers two kinds of OpenStack instances deployments. This first one, called uniform, defines a uniform distribution of the network latency between OpenStack instances. For instance, 300 ms of RTT between all the 9 OpenStack instances. The second deployment, called hierarchical, maps to a more realistic view, like in cloud computing, with groups of OpenStack instances connected through a low latency network (e.g., 3 OpenStack instances per group deployed in the same country, and accessible within 20 ms of RTT). And high latency network between groups (e.g. 150 ms of RTT between groups deployed in different countries).

\begin{figure}[H]
  \vspace{-10pt}
  \centering
  \centerline{\includegraphics[width=0.6\textwidth]{no_push/short-zones-impact-light.png}}
  \vspace{-5pt}
  \caption{Impact of the Locality on the Completion Time under Light Load – Median Time for Each Operations.}
  \vspace{-5pt}
  \label{fig:short-zones-impact-light}
\end{figure}


Figure \ref{fig:zones-impact-light}

cdf \ref{fig:zones-impact-light-cdf}
