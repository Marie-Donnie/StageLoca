My internship has been about considering the client locality in a NewSQL database. This kind of database scales while keeping the properties required to enforce proper transactions. Its entries are distributed across a cluster of nodes. What we want is to prioritize the access to one node rather the entirety of the cluster of nodes. This report will explain how we evaluated performances of different topologies and databases to pave the way towards the locality we want.

"We" designates the Discovery initiative\cite{discovery}. This is a Inria Project Lab (IPL) that involves the Stack, Avalon, Corse, Myriads and Resist teams, along with Orange and some ponctual help from Renater or RedHat. The initiative aims at implementing a fully decentralized IaaS manager, by using a revised OpenStack to make it cooperative. It studies the possiblity of infrastructures for the massively distributed cloud computing in small edge datacenters. Those infrastructures aim at avoiding the problems of datacenter scaling and limiting the latency by bringing the compute nodes closer to the users.


The Stack team, in which I worked is part of the LS2N, a ``Unité Mixte de Recherche'' (Mixed research unit) which involves the University of Nantes, Centrale Nantes, IMT Atlantique, CNRS and Inria. I personnaly worked for five months for Inria at IMT Atlantique. Inria is the French Institute for Research in Computer Science and Automation, a national research institution for computer science and applied mathematics. The IMT Atlantique Bretagne Pays de la Loire is a engineering school where most of the Stack team works.

Some work has previously been done by the Discovery initiative towards getting the compute closer to the users, and some was also ongoing in parallel. My contribution was to make Juice\footnote{\url{https://github.com/BeyondTheClouds/juice}}, a framework to deploy and evaluate OpenStack performances on different databases and topologies. It lead to a presentation at the OpenStack Summit in Vancouver\footnote{\url{https://www.openstack.org/videos/vancouver-2018/keystone-in-the-context-of-fogedge-massively-distributed-clouds}} and a blog article detailing the experiments and results\footnote{\url{https://beyondtheclouds.github.io/blog/openstack/cockroachdb/2018/06/04/evaluation-of-openstack-multi-region-keystone-deployments.html}}. The main goal of my work was to compare different kind of database under different topologies to confirm that a locality-based architecture was a good way to go.

To better explain the work done in this internship, I will first detail the context, then we will see what already exists and what solutions we considered, then I will describe the experiment plans and analyze the details to finally conclude on the future work to be done.
