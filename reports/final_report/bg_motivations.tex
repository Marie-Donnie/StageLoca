\section{Edge computing}

More and more applications use the geographic distribution of the users for a better service, such as videos streams, the internet of things related applications, or the autonomous driving applications. This kind of applications are deployed on edge infrastructures. An edge infrastructure can be defined as up to hundreds individually-managed and geo-distributed micro datacenters composed of dozens of servers, though the definitions might vary. This is the definition we, at the Discovery initiative, rely on. Edge computing is then a way to get the applications, data and services away from the centralized core to the other end of a network, closest to the users and directly in contact to ``the physical world''. It is as if there was a small brain close to the sensitive part of a neuron, rather than one centralized far away, which creates delay in response. The expected latency and bandwidth between elements may fluctuate, in particular because networks can be wired or wireless. Disconnections between sites may also occur, leading to network partitioning situations. This kind of edge infrastructure shares common basis with cloud computing, notably in terms of management. Therefore developers and operators (DevOps) of an edge infrastructure expect to find most features that made current cloud solutions successful. Unfortunately, there is currently no resource management system able to deliver all these features for the egde.

\section{OpenStack}

To study edge computing, the Discovery initiative analyzes and modifies the infrastructure manager OpenStack. OpenStack is an open-source software composed of many different services to control and manage everything needed to operate a datacenter. The services ranges from managing computing resources (e.g. virtual machines with Nova, containers with Magnum or physical machines with Ironic) to storage resources (e.g. disk volumes with Cinder, object stores with Swift), through network resources (e.g. Neutron), monitoring services (e.g. Ceilometer), benchmarking (e.g. Rally), etc.

OpenStack is one of the major solution to manage private and public cloud computing infrastructures and one that can very well be considered to handle the edge computing. Though it had been developped with the idea of being able to scale, some services, like the relational database it relies on, do not satisfy this idea (a centralized database is a single point of failure). This makes OpenStack unable, for now, to manage hundreds of small, massively distributed, datacenters. Moreover, to deploy and manage edge datacenters, OpenStack have to handle high delays. And then again, other services like the message broker become unusable under this high delays. The Discovery initiative decided to adapt OpenStack to operate simultaneously these hundreds of small edge datacenters, to avoid reinventing the wheel and building a new management system from scratch.

From a bird’s-eye view, OpenStack has two type of nodes: data nodes delivering XaaS capabilities (compute, storage, network, \dots, i.e., data plane) and control nodes executing OpenStack services (i.e., control plane). Whenever a user issues a request to OpenStack, the control plane processes it, which may potentially also affect the data plane in some manner. Preliminary studies conducted by the Discovery initiative enabled to identify different OpenStack deployment scenarios for the edge: from a fully centralized control plane, to a fully distributed control plane\cite{cherrueau:hal-01812747}.


\section{OpenStack at the Edge}

Every OpenStack service uses a relational database to save its state. This database is a single point of failure, which is a problem to achieve the goal of a massively distributed OpenStack. They are difficult to distribute, especially in a high latency context.

The Discovery initiative works on replacing the relational database by a NewSQL one, using CockroachDB, which has been inspired by Google's Spanner\cite{CRDB:HLC}. In particular, it paved the way towards a use of CockroachDB with OpenStack Keystone.

Keystone is the identity service of OpenStack. It manages the authentication of users and map them to the services they can access, using a catalog of all services deployed in the current configuration. Keystone is interesting because it does not use subqueries when making request to the database, so it is easier to ``plug'' to an other database. Moreover, when wanting to distribute OpenStack in different regions, one can either use Galera, that replicates data over the cluster, or use a distributed database such as CockroachDB in order to have all her users and services available across all regions.

Nevertheless, we have to think about how to distribute properly the data across all nodes of the clustered database. In the context of a massively distributed edge computing, we must push the idea to keep the data the closest to the users possible to avoid latency in the requests. Also, we can ask how Keystone will behave in the different control planes discussed above.
