\section{Turn on the Juice}


To evaluate if OpenStack is ready for the management of an edge infrastructure, we have built Juice\footnote{\url{https://github.com/beyondtheclouds/juice}}. Juice tests and evaluates the performance of Relational Database Management Systems (RDBMS) in the context of a distributed OpenStack with high latency network.

Juice uses EnosLib\footnote{\url{https://github.com/beyondtheclouds/enoslib}}, a library developed by the Discovery initiative to acquire resources from different infrastructures and then describe and run an experimentation workflow.

Juice uses Ansible scripts to do the required tasks,YAML files to configure the tasks and the parameters of the experiments and python files to wrap everything up. Juice install the required dependencies (python librairies, docker, chrony to synchronize the nodes, \dots), the docker container for the required RDBMS, and, depending on the parameters, benchmarking tools (see subsection \ref{subsec:benchmarking}). The RDBMS are store directly on the RAM to avoid a bias from the cache management.

Juice itself has different commands callable from a shell, such as deploy, which claim resources from Grid'5000 and configure them, emulate, that emulates a network with constraints using netem\footnote{\url{https://wiki.linuxfoundation.org/networking/netem}}, or rally, which uses the benchmarking tool of OpenStack to test different scenarios given in the parameters. Listing \ref{lst:juice-api} shows the global Juice API, along with everything that can be called using Juice.

\begin{lstlisting}[caption={Juice help command}\label{lst:juice-api}, captionpos=b]

Tool to test the performances of MariaDB, Galera and CockroachdB with
OpenStack on Grid'5000 using Enoslib

Usage:
juice [-h | --help] [-v | --version] <command> [<args>...]

Options:
    -h --help      Show this help
    -v --version   Show version number

Commands:
    deploy         Claim resources from g5k and configure them
    g5k            Claim resources on Grid'5000 (from a frontend)
    info           Show information of the actual deployment
    inventory      Generate the Ansible inventory file
    prepare        Configure the resources
    emulate        Emulate network using tc netem
    stress         Launch sysbench tests (after a deployment)
    openstack      Add OpenStack Keystone to the deployment
    rally          Benchmark the Openstack
    backup         Backup the environment
    destroy        Destroy all the running dockers (not the

Run 'juice COMMAND --help' for more information on a command
\end{lstlisting}

We use two different network interface: one to transfer metrics data to a special node that collect all of it and an other to link the nodes hosting an OpenStack instance. This is to make sure the collection of data will not be impaired by the delay we apply during the experiments.

We then developed automated python experiments, calling Juice API to implement our test plan presented in section \ref{sec:testplan}, page \pageref{sec:testplan}.

It is also worth noting that to be able to evaluate CockroachDB, we used a modified version of OpenStack oslo.db\footnote{\url{https://github.com/BeyondTheClouds/oslo.db}} and Keystone\footnote{\url{https://github.com/BeyondTheClouds/keystone}} to be able to connect the RDBMS and the service and make the necessary retries. Indeed, oslo.db has a function to describe a filter on the database backend errors to raise itself a deadlock error. This deadlock has to be intercepted in the necessary Keystone function for the retries to be executed. Also, previous work has been done to use CockroachDB as a backend for Keystone. Finally, we had to modify OpenStack Rally\footnote{\url{https://github.com/Marie-Donnie/rally/tree/fixing-regions}} because it could not manage to contact different instances of databases on different instances of OpenStack.

\subsection{OpenStack deployment}

To fully deploy an OpenStack, different approaches coexist, such as Kolla, which provides containers and tools to operate OpenStack infrastructures for production, or DevStack, which is a series of scripts to bring up an Openstack environment. DevStack uses the repositories for OpenStack services and it is then really easy to change the link to whatever fork is needed. This is why we chose this approach because it would be easier to correct any services we want if need be, which would be longer in a container approach.

The deployment of OpenStack relies on DevStack stable/pike and uses default parameters for Keystone (e.g., SQL backend, fernet token, \dots). As said previously, OpenStack, and therefore DevStack, do not natively support CockroachDB. DevStack is then tweaked to ensure Keystone connects to the right RDBMS container. The Discovery initiative made a blog article about the support of CockroachDB in Keystone\footnote{\url{https://beyondtheclouds.github.io/blog/openstack/cockroachdb/2017/12/22/a-poc-of-openstack-keystone-over-cockroachdb.html}}.

Juice default configuration for DevStack\footnote{\url{https://github.com/beyondtheclouds/juice/blob/master/ansible/roles/openstack/templates/local.conf.j2}} disables all services except Keystone and its database, but one could test any other service she wants, provided she knows how to use DevStack configuration, though.


\subsection{OpenStack benchmarking}
\label{subsec:benchmarking}

We used mainly OpenStack Rally\footnote{\url{https://github.com/openstack/rally}} to benchmark the performance of Keystone. It generates a load to evaluate how Keystone control plane behaves when scaling.
One Rally executes is load on one instance of OpenStack. This means we can either launch one Rally on the entire cluster (light load mode) or one Rally per instance (heavy load mode), which creates a huge burst. Indeed, there are two variables to configure the execution of a Rally scenario: the number of times a task is executed during the scenario and the concurrency, defining the number of parallel task executions. Thus, a scenario with a times of 100 runs one hundred iterations of the scenario by a constant load on the OpenStack instance. A concurrency of 10 specifies that the one hundred iterations are achieved by ten users in a concurrent manner.

This means that when we are executing a burst on 45 OpenStack instances (and so 45 Rallys) with a concurrency of 10 and a times of 100 charges 450 constant transactions on the RDBMS up until the 4,500 iteration are done.

In addition to Rally, Juice can also use sysbench\footnote{\url{https://github.com/akopytov/sysbench}} to benchmark the database more precisely.



\subsection{Global backup}

The execution of a Rally scenario produces a json file. The json file contains a list of entries: one for each iteration of the scenario. An entry then retains the time it takes to complete all Keystone operations involved in the Rally scenario.

Juice \verb+backup+ produces an archive with the corresponding json files to the scenarios ran for each Rally instances as well as general metrics collected over the experiments such as the CPU/RAM consumption, network communications (all stored in a influxdb archive).

If one needs to also get a backup of the studied database, she only needs to add the ansible rule in the backup file of the corresponding database (\url{https://github.com/BeyondTheClouds/juice/blob/master/ansible/roles/database/RDBMS-NAME/tasks/backup.yml}).


\section{Test plan}
\label{sec:testplan}
\subsection{The testbed: Grid'5000}
Grid’5000\footnote{\url{https://www.grid5000.fr/mediawiki/index.php/Grid5000:Home}} is a large-scale and versatile testbed for experiment-driven research in all areas of computer science, with a focus on parallel and distributed computing including Cloud, HPC and Big Data. The platform gives an access to approximately 1000 machines grouped in 30 clusters geographically distributed in 8 sites. This study uses the ecotype cluster made of 48 nodes with each:

\begin{description}
    \item[CPU:] Intel Xeon E5-2630L v4 Broadwell 1.80GHz (2 CPUs/node, 10 cores/CPU)
    \item[Memory:] 128 GB
    \item[Network:]
      \begin{itemize}
        \item[]
        \item eth0/eno1, Ethernet, configured rate: 10 Gbps, model: Intel 82599ES 10-Gigabit SFI/SFP+ Network Connection, driver: ixgbe
        \item eth1/eno2, Ethernet, configured rate: 10 Gbps, model: Intel 82599ES 10-Gigabit SFI/SFP+ Network Connection, driver: ixgbe
      \end{itemize}
\end{description}

\subsection{Rally scenarios}
\label{rally-scenarios}
Here is the complete list of rally scenarios considered in this study. Values inside the parentheses refer to the percentage of reads versus the percent of writes on the RDBMS. More information about each scenario is available in the appendix (see, Detailed Rally scenarios, page \pageref{sec:detail-rally}).

\begin{description}
    \item[keystone/authenticate-user-and-validate-token (96.46, 3.54):] Authenticate and validate a keystone token.
    \item[keystone/create-add-and-list-user-roles (96.22, 3.78):] Create user role, add it and list user roles for given user.
    \item[keystone/create-and-list-tenants (92.12, 7.88):] Create a keystone tenant with random name and list all tenants.
    \item[keystone/get-entities (91.9, 8.1):] Get instance of a tenant, user, role and service by id’s. An ephemeral tenant, user, and role are each created. By default, fetches the ’keystone’ service.
    \item[keystone/create-user-update-password (89.79, 10.21):] Create user and update password for that user.
    \item[keystone/create-user-set-enabled-and-delete (91.07, 8.93):] Create a keystone user, enable or disable it, and delete it.
    \item[keystone/create-and-list-users (92.05, 7.95):] Create a keystone user with random name and list all users.
\end{description}

As previously said, Rally can be ran in a light load mode or heavy load mode for every scenarios. Figures \ref{fig:rally-light} and \ref{fig:rally-high} show the principle on light and heavy load on three nodes: the white ermine is the Rally mascot and the red logo is Openstack's.


\begin{figure}[!h]\centering
  \begin {minipage}{0.3\textwidth}                        \centerline{\includegraphics[width=\linewidth]{no_push/rally-light.png}}
    \vspace{-20pt}
    \caption{Rally light load mode}
    \label{fig:rally-light}
  \end{minipage}
  \begin{minipage}{0.3\textwidth}                   \centerline{\includegraphics[width=\linewidth]{no_push/rally-high.png}}
    \vspace{-10pt}
    \caption{Rally heavy load mode}
    \label{fig:rally-high}
  \end{minipage}
\end{figure}

\subsection{Multiple OpenStack instances deployment with the databases}


\subsubsection{MariaDB and Keystone: a single MariaDB}


Figure \ref{fig:key-dep-centralized} depicts the deployment for MariaDB. MariaDB is a centralized RDBMS and thus, the Keystone backend is centralized in the first OpenStack instance.

\begin{figure}
  \begin{center}
    \includegraphics[width=0.5\textwidth]{no_push/keystone-centralized.png}
  \end{center}
  \vspace{-20pt}
  \caption{Keystone deployment with a centralized MariaDB.}
  \vspace{-15pt}
  \label{fig:key-dep-centralized}
\end{figure}

 Other Keystones from the other OpenStack nodes refers to the backend of the first instance. The red logo is an OpenStack instance, the turtle is the Keystone logo.

This kind of deployment is easy to set up since you only need to have the IP address of the node storing MariaDB to add any other node, so the nodes configurations are all the same. MariaDB also has the advantage of having a lot of work done on the performance under different types of workloads.

But this kind of deployment comes with -at least- two possible limitations. First, a centralized RDBMS is a single point of failure that makes all OpenStack instances unusable if it crashes. Second, a network disconnection of, for example, one of the ``bottom'' OpenStack instance with the one that stores MariaDB makes the separated node unusable.



\subsubsection{Galera in multi-master replication mode and Keystone: over replicated Galera}
The figure \ref{fig:key-dep-galera} depicts the deployment for Galera. Galera synchronizes multiple MariaDB databases in an active/active (multi-master) fashion, as previously explained in section \ref{sec:galera-cluster}.

\begin{figure}[h]
  \vspace{-10pt}
  \begin{center}
    \includegraphics[width=0.5\textwidth]{no_push/keystone-galera.png}
  \end{center}
  \vspace{-20pt}
  \caption{Keystone deployment with synchronized MariaDB instances thanks to Galera.}
  \vspace{-12pt}
  \label{fig:key-dep-galera}
\end{figure}

 Thus Keystone’s backend of every OpenStack instance is replicated between all nodes, which allows reads and writes on any instances. The figure semantic is the same as in figure \ref{fig:key-dep-centralized}, but with the added synchronization between the replicated MariaDB thanks to Galera (blue ring).

We use a multi-master topology, which means reads and writes can be done on any nodes. We also use the \verb+wsrep_sync_wait+ at the maximum level to ensure causality checks on \verb+READ+, \verb+UPDATE+, \verb+DELETE+, \verb+INSERT+, \verb+REPLACE+ and \verb+SHOW+ operations. This means there is a check to see if the node is fully synchronized before these operations are executed.

Regarding possible limitations, few rumors\footnote{\url{http://www.fromdual.com/limitations-of-galera-cluster}} stick to Galera. First, synchronization may suffer from high latency networks. Maybe not for the performances since it does not wait for the answers of the other nodes to commit a transaction, but on the consistency of the database if the logical time is not well defined, as we previously thought (see section \ref{sec:galera-cluster}). Second, contention during writes on the database may limit its scalability.



\subsubsection{CockroachDB and Keystone: over distributed CockroachDB}
Figure \ref{fig:key-dep-crdb} depicts the deployment for CockroachDB. In this deployment, each OpenStack instance has its own Keystone.

\begin{figure}[h]
  \vspace{-10pt}
  \begin{center}
    \includegraphics[width=0.5\textwidth]{no_push/keystone-cockroachdb.png}
  \end{center}
  \vspace{-20pt}
  \caption{Keystone deployment with CockroachDB.}
  \vspace{-12pt}
  \label{fig:key-dep-crdb}
\end{figure}

The backend is distributed through key-value stores on every OpenStack instance. Meaning, the data a Keystone is sought for is not necessarily in its local key-value store.

CockroachDB is relatively new and we know a few about its limitations, but first, CockroachDB may suffer from high network latency even during reads if the data is located on another node. Second, as Galera, transaction contention may dramatically slow down the overall execution.

However, CockroachDB offers locality option (replicas zone) to drive the selection of key-value stores during writes and replication. Thanks to this option, we managed to try a topology that considered locality.


\subsubsection{CockroachDB and Keystone: over distributed CockroachDB, using replication zones}
\label{subsubsec:replication-zones}
 it is be possible to mitigate the impact of latency by ensuring that writes happen close the OpenStack operator. Indeed, we can choose to refer to three of the instances as the same region (with \verb+--locality=region=REGION-NAME,datacenter=NODE-NAME+) and then specify that the Keystone table should be only on this region. For example, to specify one replica per region:
\begin{verbatim}
constraints: {
"+region=REGION-NAME-1": 1,
"+region=REGION-NAME-2": 1,
"+region=REGION-NAME-3": 1
}
\end{verbatim}

\begin{figure}
  \begin{center}
    \includegraphics[width=0.5\textwidth]{no_push/locality.png}
  \end{center}
  \vspace{-20pt}
  \caption{Keystone deployment over CockroachDB using replication zones, three replicas on the first group.}
  \vspace{-15pt}
  \label{fig:locality}
\end{figure}


Figure \ref{fig:locality} represents the final type of deployment we used. The OpenStack logos represents an instance of OpenStack and Keystone. All of them also have a CockroachDB instance, represented by the CockroachDB logo (the green cockroach). The database has a Keystone table, which gets all Keystone data, distributed using the replication zones. The replicas are represented by the non-transparent logo, the other are instances of CockroachDB without this table.

We distributed 9 nodes into groups of 3. This 3 groups represents 3 continents, each having 3 different datacenters in them. They are separated by a network delay of 300 ms of RTT, and 20 ms of RTT between each datacenters (nodes).

We then chose to force the replicas in different ``continents'' (groups). Rally is located in the ``first'' group. First, we distributed the replicas to the entire cluster, which means one replicas per group. Then, we force CockroachDB to keep 2 replicas (counting the leaseholder) to be in the same group (the first), so the quorum can be achieved in this group, without having to pay the cost of the latency towards another group. The third option is to get all the replicas in the first group, to not only have the quorum in this group, but also all the data and replicas in the same location. In both the second and third cases, we expect to mitigate the latency effects by considering the client locality in the database.


\subsection{Parameters variations}
\label{subsec:param}
\subsubsection{Number of OpenStack instances}

The OpenStack size defines the number of OpenStack instances deployed for an experiment. It varies between 3, 9 and 45. A value of 3, means Juice deploys OpenStack on three different nodes, 9 on nine different nodes, \dots The value of 45 comes from the maximum number of nodes available on the ecotype Grid’5000 cluster, but Juice is not limited itself.


Experiments that test the impact of the number of OpenStack instances consider a LAN link between each OpenStack instances.

\subsubsection{Delay}

The delay defines the network latency between two OpenStack instances. It is expressed in terms of the time for a signal sent from one node to reach another, (i.e., a value of 50 stands for 100 ms of Round-Trip Time, 150 is 300 ms of RTT). The 0 value stands for LAN speed which is approximately 0.08 ms of RTT on the ecotype Grid’5000 cluster (10 Gbps card).

Juice applies theses network latencies with tc netem. Note that, as previously said, juice applies tc rules on network interfaces that are dedicated to the RDBMS communications. Thus, metrics collection and other network traffics are not limited.

Experiments that test the impact of the network latency are done with 9 OpenStack instances. They make the delay varies by applying traffic shaping homogeneously between the 9 OpenStack instances.

\subsection{How we make the analysis}

As previously said, we call Juice API in python files, one for each experiments. We use Execo Engine ParamSweeper\footnote{\url{http://execo.gforge.inria.fr/doc/latest-stable/execo_engine.html}} to shuffle the parameters we want to test. We give lists of parameters and ParamSweeper run the experiment for the combinations of parameters, e.g. giving for the RDBMS the list [MariaDB] and for the latencies [0, 50], it will run the scenarios with MariaDB as the RDBMS for 0 and 50 ms latency (once again, the given latency is the time for a signal to go from the source to destination, so we have to double it for RTT). We give all the configuration directly and make a deep copy to change whatever we need in the process.

The main point of having different experiment files is to be able to run experiments in parallel when the number of nodes allow it (the cluster we are testing has 48 nodes and we always need x nodes for the OpenStack instances plus one for the monitoring, so at maximum 46 for the 45 nodes experiments). We also made a different branch for the locality experiment because it required to change more than one file. Since all experiments take a lot of time (several hours to run all the scenarios, depending on the configuration, and CockroachDB especially had difficulties to delete the context Rally created), it was pretty handy to be able to run some experiments in parallel. And it was particularly helpful because we tested different test plans before deciding a good one and we also tested CockroachDB first major version before the second version was available and working for our context.

We have several ``main'' variables, with the name corresponding to the parameters: \verb+DATABASES+ taking the values ['mariadb', 'galera', 'cockroachdb'], \verb+CLUSTER_SIZES+, with the values [3, 9, 45], \verb+DELAYS+, with the values [0, 50, 150], and finally \verb+BURST+ taking [False, True] for the load mode (respectively light and heavy).

\begin{description}
\item[cluster-size-impact\footnotemark]\footnotetext{\url{https://github.com/BeyondTheClouds/juice/blob/master/experiments/cluster-size-impact.py}} This experiment is about the impact of the number of OpenStack instances, so the change is in the \verb+CLUSTER_SIZES = [3, 9, 45]+.
  \item[latency-impact\footnotemark]\footnotetext{\url{https://github.com/BeyondTheClouds/juice/blob/master/experiments/latency-impact.py}} This experiment is about testing the impact of the delay between the cluster nodes, so the main change is in \verb+DELAYS = [0, 50, 150]+
\item[regions\footnote{\url{https://github.com/BeyondTheClouds/juice/blob/regions/experiments.py}}] The file is different since neither the delays nor the cluster size changes. The varying parameter here is the number of replicas in the first group (see subsection \ref{subsubsec:replication-zones}) \verb+REPLICAS_QUORUM = [3, 2, 1]+. 3 means all replicas are in the first zone, 2 means 2 are in the first group and one in the second and 1 means replicas are distributed across the groups.
  \item[read-write-ratio\footnotemark]\footnotetext{\url{https://github.com/BeyondTheClouds/juice/blob/master/experiments/cluster-size-impact.py}} This is the experiment that measured the reads and writes ratio for each scenario.
\end{description}
