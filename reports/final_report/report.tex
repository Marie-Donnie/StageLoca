
\documentclass[a4paper, 10pt, titlepage]{report}

%% Mandatory stuff
\usepackage[utf8]{inputenc}
\usepackage[T1]{fontenc}
%% --


%% Color package for colored links
\usepackage[usenames,dvipsnames]{xcolor}
\definecolor{blues}{RGB}{153,50,204}
%% --


%% Code listings
\usepackage{listings}
% UTF8 compatibility options
% UTF8 chars are OK in code comment
% but not elsewhere
\lstset{extendedchars=false, texcl}
% Eye candy

%% For temporary redefining the geometry of titlepage.
\usepackage{geometry}

\lstdefinestyle{python}{
  language=python,
  basicstyle=\footnotesize\rmfamily,
  identifierstyle=,
  commentstyle=\itshape,
  keywordstyle=,
  stringstyle=,
  showspaces=false,
  showstringspaces=false,
  tabsize=3,
  breaklines=true
}

\newcommand{\code}[1]{\lstinline!#1!}

\lstset{style=python}

%% Theorems environments
\usepackage{amsmath}
\usepackage{amsthm}

%% Power to the tikz!
\usepackage{tikz}
%\usepackage{graphicx}
%% --

%% Page layout
\usepackage{makeidx}
\makeindex

%% Plain page headings
\pagestyle{headings}

%% Smaller space between heading and matter
\setlength\headsep{0.5in}
 %% --

%% Larger page
%%\usepackage[left=4cm,right=4cm,top=5cm,bottom=4cm]{geometry}

%% Adjust toc depth
\setcounter{tocdepth}{4}

%% Dotted toc
\usepackage{tocloft}
\renewcommand{\cftsecleader}{\cftdotfill{\cftdotsep}}

%% Figures where I want them
\usepackage{float}

%% Nice caption
\usepackage[font={footnotesize}]{caption}

%% Footnote where they belong
\usepackage[bottom]{footmisc}

%% Comments
\newcommand{\comment}[1]{}


%% Removes the ``chapter'' heading without the chapter number
%% \usepackage{titlesec}
%% \titleformat{\chapter}[display]
%%   {\normalfont\bfseries}{}{0pt}{\Huge \thechapter\ - }

%% Image in title
\usepackage{graphicx}

%% Images wrapped in text
\usepackage{wrapfig}

%% Subfigures
\usepackage{subcaption}

%% Bibliography
\usepackage[style=numeric]{biblatex}

\addbibresource{biblio.bib}

%% Hyperlinks for references, in color-O-vision and nice for long urls
%% MUST BE THE LAST PACKAGE
\PassOptionsToPackage{hyphens}{url}
\usepackage[pdftex,%
  pdfauthor={Marie Delavergne},%
  pdftitle={Report for  Master 2 ALMA},%
  pdfsubject={A middleware for a NewSQL database that reflects the client application locality},%
  colorlinks,%
  linkcolor=blues,%
  urlcolor=blues,%
  plainpages=false]{hyperref}
%% --


\begin{document}
%
%% \maketitle
\newgeometry{bottom=2.5cm}

\begin{titlepage}
  \centering
      {\scshape\LARGE\bfseries Considering the client locality in the context of Edge Computing \par}
      \vspace{1cm}
             {\Large - Internship Report for Master 2 ALMA -\par}
             {\Large January--June 2018 \par}
             \vspace{1.5cm}
               	    {\Large\itshape Marie Delavergne\par}

	            \vfill

                    % Bottom of the page

                    {\large Supervisors: Ronan-Alexandre Cherrueau (Inria)\\
                      \hspace{3.45cm}Achour Mostefaoui (University of Nantes)\par}
                          \vspace{2cm}
                    {\large Stack - DAPI - LS2N - Inria - Discovery Initiative\par}
	            {\large University of Nantes\par}
                          \vspace{0.5cm}
                    \begin{figure}[!h]\centering
                      \begin {minipage}{0.3\textwidth}                        \centerline{\includegraphics[width=\linewidth]{no_push/inr_logo_eng_rouge.png}}
                        \label{Fig:inria}
                      \end{minipage}
                      \begin{minipage}{0.3\textwidth}                   \centerline{\includegraphics[width=\linewidth]{no_push/logo_un2012quadri_larg40.png}}
                        \label{Fig:univnantes}
                      \end{minipage}


                    \end{figure}


\end{titlepage}
\restoregeometry

\clearpage
\tableofcontents
\newpage

\section*{Aknowledgement}

I would like to thank first Ronan-Alexandre for his supervision, knowledge, work, help, support and patience. The Discovery and Stack team for welcoming me and helping me. Thanks to Adrien and Inria for trusting me and sending me to Vancouver. To Achour for his teachings about the consensuses and logical time. To Olivier Grasset and the ESA for the inspiration for the name of my framework. And of course Florent for his help and support.

\newpage

\section*{Abstract}

An ever growing number of applications requires a wide distribution over the globe of computing resources, such as the Internet of Things (IoT). To answer this need, we require Edge infrastructures fully automated, using a resource management system able to provide everything needed to handle the difficulties. These are managing hundreds of datacenters with dozen of servers, heterogeneous wide networks with possible disconnections and partitions. Building this kind of systems from scratch is too much of a challenge, thus we think it is better to change an existing, widely used IaaS manager in order to satisfy the requirements.
OpenStack is made to be scalable, except for some components such as the centralized database, which represents a single point of failure. NewSQL databases, such as CockroachDB, are made to scale under OLTP while complying the relational model which OpenStack uses. While an Edge infrastructure manages high latency connections, we think that is really important to consider the locality of the client to reduce network latency.
However, we do not know how would OpenStack behave in an Edge context, or the concrete impact of considering the locality.
We present here an evaluation of the performances of Keystone, an OpenStack service, over three RDBMS, namely MariaDB, MariaDB Galera Cluster and CockroachDB. Galera is the clustering solution used by MariaDB.
We found that MariaDB Galera Cluster handles pretty well the high delays but not the number of nodes, and conversely for CockroachDB. But more important, we confirmed that CockroachDB has excellent performances when we consider locality.
We think that there is still more work to do regarding the parametrization of the locality to refine it up to the query.


\section*{Résumé}

De plus en plus d'applications utilisent une distribution globale de ressources de calcul, telles que l'Internet des Objets. Pour répondre à ce besoin, il est nécessaire que les infrastructures d'\emph{informatique en périphérie} soient entièrement automatisées en utilisant un système de gestion de ressources capable de remplir tous les besoins de ces infrastructures. Elles nécessitent de faire face à plusieurs difficultés comme gérer des centaines de centres de données comportant eux-mêmes des dizaines de serveurs, gérer le fait d'avoir des réseaux hétérogènes avec de possibles déconnexions et des partitions. Construire un tel système de zéro est bien trop difficile et nous pensons alors qu'il est plus intelligent de modifier un gestionaire d'Infrastructure-en-tant-que-Service pour satisfaire ces besoins.
OpenStack a été fait pour pouvoir passer à l'échelle, sauf certains composants tels que la base de données centralisée, qui représente un point unique de défaillance. Les bases de données NewSQL telles que CockroachDB ont été faites pour gérer du traitement transactionnel en ligne tout en fournissant une interface satisfaisant le modèle relationnel qu'utilise OpenStack. Bien qu'une infrastructure en périphérie doivent gérer des connections à forte latences, nous pensons qu'il est nécessaire de prendre en compte la localité du client pour réduire les latences lors des requêtes.
Cependant, nous ignorons comment OpenStack se comporterait réellement dans un contexte d'informatique en périphérie, ou l'impact concret de cette prise en compte de la localité.
Nous présentons ici une évaluation des performances de Keystone, un service d'OpenStack, sur trois systèmes de gestion de base de données relationnelles : MariaDB, MariaDB Galera Cluster et CockroachDB. Galera est la solution de grappe de serveurs utilisée par MariaDB.
Nos résultats montrent que MariaDB Galera Cluster gère correctement des fortes latences réseaux, mais pas le nombre de serveurs, et inversement pour CockroachDB. Mais plus important, nous avons confirmé que CockroachDB obtient d'excellentes performances lorsque nous prenons en compte la localité.
Nous pensons qu'il y a toujours du travail à effectuer concernant la paramétrisation de la localité pour qu'elle soit aussi fine qu'au niveau d'une requête.


\addcontentsline{toc}{chapter}{\textbf{Introduction}}
\chapter*{Introduction}

Consider a bank. In the past, every transaction was recorded on paper, everything was computed manually and people working in the bank probably made mistakes. Nowadays, everything is digital and the bank software can not be fallible. The software must also store all the information about its clients, accounts, transactions, \dots.

In the digital word, this information is stored in databases, which are exactly organized collections of data. The databases can store all sorts of data, like words, numbers, or even texts. To store these data, they traditionally use different tables for different information, linked and identified by an unique identifier (ID). As we can see on tables \ref{tab:clientname} and \ref{tab:accountbalance}, the ID is the account's number and it is used to point a client name to her account balance. A single row refers to the same ID. In this example, Scrooge McDuck, with the account number 1, has a balance of \$300,000,000,000,000. Donald Duck, on the other hand, only has \$5 on his account numbered 2.
\begin{table}[H]
  \begin{minipage}[b]{0.45\linewidth}\centering
\begin{tabular}{|c|c|c|}
\hline
ID & First Name & Last Name \\
\hline
1 & Scrooge & McDuck \\
2 & Donald & Duck \\
\hline
\end{tabular}
\caption{Client name table}
\label{tab:clientname}
  \end{minipage}
\begin{minipage}[b]{0.45\linewidth}\centering

\begin{tabular}{|c|c|}
\hline
ID & Balance \\
\hline
1 & 300,000,000,000,000 \\
2 & 5 \\
\hline
\end{tabular}
\caption{Account balance table}
\label{tab:accountbalance}
\end{minipage}
\end{table}


My internship has been about considering the client locality in a NewSQL database. This kind of database scales while keeping the properties required to enforce proper transactions. Its entries are distributed across a cluster of nodes. What we want is to prioritize the access to one node rather the entirety of the cluster of nodes. This report will explain how we evaluated performances of different topologies and databases to pave the way towards the locality we want.

"We" designates the Discovery initiative\cite{discovery}. This is a Inria Project Lab (IPL) that involves the Stack, Avalon, Corse, Myriads and Resist teams, along with Orange and some ponctual help from Renater or RedHat. The initiative aims at implementing a fully decentralized IaaS manager, by using a revised OpenStack to make it cooperative. It studies the possiblity of infrastructures for the massively distributed cloud computing in small edge datacenters. Those infrastructures aim at avoiding the problems of datacenter scaling and limiting the latency by bringing the compute nodes closer to the users.


The Stack team, in which I worked is part of the LS2N, a ``Unité Mixte de Recherche'' (Mixed research unit) which involves the University of Nantes, Centrale Nantes, IMT Atlantique, CNRS and Inria. I personnaly worked for five months for Inria at IMT Atlantique. Inria is the French Institute for Research in Computer Science and Automation, a national research institution for computer science and applied mathematics. The IMT Atlantique Bretagne Pays de la Loire is a engineering school where most of the Stack team works.

Some work has previously been done by the Discovery initiative towards getting the compute closer to the users, and some was also ongoing in parallel. My contribution was to make Juice\footnote{\url{https://github.com/BeyondTheClouds/juice}}, a framework to deploy and evaluate OpenStack performances on different databases and topologies. It lead to a presentation at the OpenStack Summit in Vancouver\footnote{\url{https://www.openstack.org/videos/vancouver-2018/keystone-in-the-context-of-fogedge-massively-distributed-clouds}} and a blog article detailing the experiments and results\footnote{\url{https://beyondtheclouds.github.io/blog/openstack/cockroachdb/2018/06/04/evaluation-of-openstack-multi-region-keystone-deployments.html}}. The main goal of my work was to compare different kind of database under different topologies to confirm that a locality-based architecture was a good way to go.

To better explain the work done in this internship, I will first detail the context, then we will see what already exists and what solutions we considered, then I will describe the experiment plans and analyze the details to finally conclude on the future work to be done.



\chapter{Openstack at the Edge}
\label{chap:os-edge}

\subsection{Edge computing}


\subsection{Problem}



\chapter{Relational database management systems considered}
\label{chap:rdmbs-considered}
Two databases were considered: one distributed, CockroachDB, and the other replicated, MariaDB Galera Cluster. We also used MariaDB as a control for the other databases, keeping in mind that it is a single point of failure, so we can not really use it in our context of edge computing.

\subsection{MariaDB Galera Cluster}

MariaDB, an open-source fork of MySQL, uses Galera Cluster as a synchronous multi-master cluster. It means that all nodes in the cluster are masters, with an active-active synchronous replication, so it is possible to read or write on every node at any time.

\begin{figure}[H]
  \vspace{-10pt}
  \centering
  \centerline{\includegraphics[width=1\textwidth]{no_push/galera-2.pdf}}
  \vspace{-5pt}
  \caption{MariaDB Galera Cluster}
  \vspace{-5pt}
  \label{fig:MGC}
\end{figure}

To say it in a more understandable way, MariaDB Galera Cluster allows to have the exact same database on every nodes thanks to a synchronous replication.

To dive more into details, each time a transaction is requested by a client on a node, it is processed as usual until the client issues a commit. The process is stopped and all changes made to the database in the transaction are collected in a ``write-set'', along with the primary keys of the changed rows. The write-set is then broadcasted to every nodes. It then undergoes a deterministic certification test which uses the given primary keys. It actually checks all transactions between the current transaction and the last successful one to determine whether the primary keys involved conflicts between each others.

If the check fails, the whole transaction is rollbacked, and if it succeeds it is applied on all nodes then committed.

\begin{figure}[H]
  \vspace{-10pt}
  \centering
  \centerline{\includegraphics[width=0.85\textwidth]{no_push/certification_based_replication.png}}
  \vspace{-5pt}
  \caption{Certification Based Replication from \url{http://galeracluster.com} }
  \vspace{-5pt}
  \label{fig:certificationcommit}
\end{figure}


This is pretty efficient since it only needs a broadcast to make the replication, which means. But this means that when it fails, the entire transaction must be retried and so it may lead to more conflicts and even deadlocks.

It also have advantages like granting there will be no data loss when nodes crash, providing a high availability across the cluster.

Galera Cluster uses a virtual synchrony that unifies the data delivery, providing a formalism for its semantics. Since this does not guarantee temporal synchrony, it also implements a flow control to keep nodes synchronized.

The multi-master replication allows only read-repeatable isolation at best for transactions, but more will be said on the subject in section \ref{sec:isolation}, \pageref{sec:isolation}.


\subsection{CockroachDB}
CockroachDB is a NewSQL database that uses the Raft protocol (an alternative version to Lamport's Paxos consensus protocol).
It uses the SQL API to enable SQL requests on every nodes. These requests are translated to key-value operations and -if needed- distributed across the cluster.

CockroachDB implements a single, monolithic sorted map for the keys and values stored, as seen in \ref{fig:cockroachdb}. This map is divided in ranges, which are continuous chunks of this map, with every key being in a single range, so the ranges will not overlap. Each ranges are then replicated (three replicas per range) and finally distributed across the cluster nodes.\cite{CRDB:automatedoperations}

\begin{figure}[H]
  \vspace{-10pt}
  \centering
  \centerline{\includegraphics[width=1\textwidth]{no_push/cockroachdb.png}}
  \vspace{-5pt}
  \caption{CockroachDB ranges, replicas and leaseholders}
  \vspace{-5pt}
  \label{fig:cockroachdb}
\end{figure}

One of the replicas acts as the leaseholder, a sort of leader that coordinates all reads and writes for the range. A read only requires the leaseholder.
When a write is requested, the leaseholder prepares to append it to its log, forward the request to the replicas and when the quorum is achieved, commit the change by add it in the log. The quorum is an agreement from two out of the three replicas to make the change.

\begin{figure}[H]
  \vspace{-10pt}
  \centering
  \centerline{\includegraphics[width=0.85\textwidth]{no_push/commit-cockroach.png}}
  \vspace{-5pt}
  \caption{CockroachDB commit: only a quorum of 2 is required to commit }
  \vspace{-5pt}
  \label{fig:cockroachdb-commit}
\end{figure}

To implement an SQL API, Cockroach uses an encoding tool to go from SQL data to a key-value store\cite{CRDB:mapKV}. As an example, a bit of code:
\begin{verbatim}
CREATE TABLE test (
      key       INT PRIMARY KEY,
      floatVal  FLOAT,
      stringVal STRING
)

INSERT INTO test VALUES (10, 4.5, "hello”)
\end{verbatim}

This row would be stored as:

\begin{center}
\begin{tabular}{|l|l|}
\hline
Key & Value\\
\hline
/test/10/floatVal & 4.5\\
/test/10/stringVal & "hello"\\
\hline
\end{tabular}
\end{center}

Here, \verb~/test/~ is a placeholder for the table ID and the \verb~/*Val~ are placeholders for the column ID used in CockroachDB. Each non-primary key column are stored under a separate key that is prefixed by the primary key (following the table ID) and suffixed by the column ID.

\subsubsection{Raft}

Raft was implemented to be a more understandable and easier to learn version of Paxos\cite{DBLP:conf/usenix/OngaroO14}. Though it is safe, available and efficient for common operations, when a choice had to be made, it was always favoring understandability.


It uses a consensus algorithm in the context of Replicated State Machines. In this context, state machines on a set of nodes compute copies of the same state and continue their work even if some nodes are down. In the context of CockroachDB, the state machines are the ranges and so there is one consensus group per range. This means each nodes participates in thousands of consensus group. So they introduced a new layer they called Multiraft to handle this: it manages the communication by treating a node with all its ranges as a group. That is, each pair of nodes only needs to exchange heartbeat messages once per tick to know if the ranges on these nodes are available\cite{CRDB:multiraft}.

Raft safety properties under non-Byzantines conditions (such as network delays, partitions, packet loss, etc.) have been formally specified and proven. It also guarantees availability if a majority $(\frac{n}{2} + 1)$ of servers are operational and communicate with each other and consistency in the logs independently of the timing (delays and faulty clocks only cause availability problems). Finally, the time to complete a command is only as long as it is necessary to receive responses from the majority of the cluster.

To dive deeper into the algorithm, a few things must be explained more. At any given time, a server can be either: a \emph{leader}, a \emph{follower} or a \emph{candidate}. Only one \emph{leader} can be elected at any time. He sends periodic heartbeat to followers as soon as he is elected to prevent further election. The leader main roll is to accept log entries from the clients and replicate them to other servers. The \emph{followers} are "passive", they respond to requests from leaders and candidates. Finally, \emph{candidates} are used to elect a new leader.

Time is divided into \emph{terms} of arbitrary length. These terms are numbered with consecutive integers, increasing monotonically and all servers (ranges for CockroachDB) stores the current term. Each term begins with an election with one or more candidate try to become leader. If an election ends with a split vote, the term ends and a new one begins. This serves as a logical clock across the cluster: if two servers communicate and don't have the same term value, they take the largest one as the current. If a candidate or leader discovers this way that his term has ended, he becomes a follower. If the server receives a request with an older term, it rejects the request.

The election runs as follows: if a leader fails or is disconnected, servers know it because they don't receive the heartbeat before the election timeout, as explained before, so a new leader is elected. When a follower comes to the election timeout, it increments its current term, becomes a candidate, votes for itself and sends a RequestVote RPC to every server in the cluster. A candidate wins the election if it receives votes from a majority of servers for the same term and the followers vote for one candidate on a first-come first-served basis. If a candidate gets a signal from a leader and the term received is at least as large as the candidate current term, the candidate returns to the follower status. There is a mechanism to prevent different rounds of election due to too many followers becoming candidates at the same time: the election timeouts are chosen randomly by the servers from a fixed interval. A follower denies a candidature its log is more up-to-date than the candidate's.

\begin{figure}[H]
  \vspace{-10pt}
  \centering
  \centerline{\includegraphics[width=0.7\textwidth]{no_push/raft_logs.png}}
  \vspace{-5pt}
  \caption{Raft logs\cite{DBLP:conf/usenix/OngaroO14}}
  \vspace{-5pt}
  \label{fig:raft-log}
\end{figure}


An entry of a log stores a command (request), a term number and an integer identifying its position in the log when the entry was received by the leader. When receiving a new entry, the leader appends the request to its log as a new entry. It then sends the order to append the entries to the other servers to replicate it (using an AppendEntries RPC). The leader then "chooses" when an entry is committed, i.e. when an entry has replicated to a majority of servers. This operation also commits all previous entries in the leader's log, including entries from former leader. The leader maintains a nextIndex for each follower so he knows which entry it will send. RPCs are idempotent; a request that includes previously recorded log is ignored.


CockroachDB leaseholder is pretty similar to Raft's leader, but it bypasses the algorithm for reads. Indeed, if the writes have been committed, they already have achieved consensus and so this operation does not require another consensus and the leaseholder handles the read by itself. It is also worth adding that CockroachDB \emph{attempts} to elect a leasholder who is also a Raft group leader to optimize the speed of writes, but this means it is not mandatory\cite{CRDB:replication-layer}.

\label{sec:isolation}
\subsection{Isolation}


There are different anomalies that can occur in Snapshot Isolation that can't happen in Serializable Isolation, and we will detail them in the following sections.

\subsubsection{Write skews}


Difference between Serializable and Snapshot Isolations from \href{https://www.cockroachlabs.com/blog/what-write-skew-looks-like/}{Cockroach Labs}:


Consider the following transactions:
\begin{table}[H]
  \centering
  \begin{tabular}{ c }
    $P:   y \gets x$\\
    $Q:   x \gets y$\\
  \end{tabular}
\end{table}


There are two possible serial executions: $P$ then $Q$ OR $Q$ then $P$
\begin{table}[H]
  \centering
  \begin{tabular}{ r | l }
    $P$ then $Q$ & $Q$ then $P$ \\
        \hline
    (P) $y \gets x$ & $x \gets y$ (Q) \\
    (Q) $x \gets y$ & $y \gets x$ (P) \\
    $x = y$ & $x = y$ \\
  \end{tabular}
\end{table}
Which, more developped, can be read as:
\begin{table}[H]
  \centering
  \begin{tabular}{ r | l }
    $P$ then $Q$ & $Q$ then $P$ \\
        \hline
    $R_P(x_0)$ &  $R_Q(y_0)$ \\
    $W_P(y_1, x_0)$ & $W_Q(x_1, y_0)$ \\
    $R_Q(y_1)$ &  $R_P(x_1)$ \\
    $W_Q(x_1, y_1)$ & $W_P(y_1, x_1)$ \\
    $x = y$ & $x = y$ \\
  \end{tabular}
\end{table}

Snapshot isolation allows another execution:

\begin{table}[H]
  \centering
  \begin{tabular}{ r  l }
    $P$ & $Q$ \\
    1. $R_P(x_0)$ & 2. $R_Q(y_0)$ \\
    3. $W_P(y_1, x_0)$ & 4. $W_Q(x_1, y_0)$ \\
  \end{tabular}
  \\
  $x \neq y$, values have been swapped
\end{table}

Indeed, each transaction kept a consistent view of the database and the writes were not on the same value, so no overlaping of the transaction write set occured, then no retries were triggered.

Another example, related to the application and the semantic of it. Consider the following table:

\begin{table}[H]
  \centering
  \begin{tabular}{ c | c }
    id & counter \\
        \hline
    0 & 1 \\
    1 & 1 \\
  \end{tabular}
\end{table}

The application using this table has a rule where at least one counter must be non-zero. Now, we can think rapidly of two non-concurrent transactions that will be problematic in a non serializable isolation:

\begin{table}[H]
  \centering
  \begin{tabular}{ c }
    $P:   0 \gets 0$\\
    $Q:   1 \gets 0$\\
  \end{tabular}
\end{table}

When executed simultaneously (which is correct in term of concurrency, they are writing on different rows), these transactions will violate the application's requirement.

\subsubsection{Read-Only transaction anomaly}


The read-only transactions have been assumed to execute serializability before Fekete, O'Niel and O'Neil proved this is not always true\cite{DBLP:journals/sigmod/FeketeOO04}.
To be perfectly honest, this example involves writes, of course. But not IN the transaction where the error appears. Suppose that we have a bank table.

\begin{table}[H]
  \centering
  \begin{tabular}{ c | c }
    type & balance \\
    \hline
    checkings (C) & 0 \\
    savings (S) & 0 \\
  \end{tabular}
\end{table}

Once again, consider a set of transactions:
\begin{table}[H]
  \centering
  \begin{tabular}{ l }
    $T_1:   S \gets S + 20 $\\
    $T_2:   C \gets C -10 ; if (C+S) <= 0, C \gets C -1 $\\
    $T_3:   C ; S$\\
  \end{tabular}
\end{table}

We can have this execution, since the writes are not concurrent:

\begin{table}[H]
  \centering
  \begin{tabular}{ r | c | l }
    $T_1$ & $T_2$ & $T_3$ \\
    & 1. $R_2(C_0,0)$ & \\
    & 2. $R_2(S_0,0)$ & \\
    3. $R_1(S_0,0)$ & & \\
    4. $W_1(S_1,20)$ & & \\
    5. $C_1$ & & \\
    & & 6. $R_3(C_0,0)$ \\
    & & 7. $R_3(S_1,20)$ \\
    & & 8. $C_3$\\
    & 9. $W_2(C_1,-11)$ & \\
    & 10. $C_2$ & \\
  \end{tabular}
\end{table}



\chapter{How to compare relational database management systems}
\label{chap:experiments}
\section{Turn on the Juice}


To evaluate if OpenStack is ready for the management of an edge infrastructure, we have built Juice\footnote{\url{https://github.com/beyondtheclouds/juice}}. Juice tests and evaluates the performance of Relational Database Management Systems (RDBMS) in the context of a distributed OpenStack with high latency network.

Juice uses EnosLib\footnote{\url{https://github.com/beyondtheclouds/enoslib}}, a library developed by the Discovery initiative to acquire resources from different infrastructures and then describe and run an experimentation workflow.

Juice uses Ansible scripts to do the required tasks and YAML files to configure the tasks and the parameters of the experiments. Juice install the required dependencies (python librairies, docker, chrony to synchronize the nodes, \dots), the docker for the required DBMS, and, depending on the parameters, benchmarking tools (see subsection \ref{subsec:benchmarking}). The DBMS are store directly on the RAM.

Juice itself has different commands callable from a shell, such as deploy, which claim resources from Grid'5000 and configure them, emulate, that emulates a network with constraints using netem\footnote{\url{https://wiki.linuxfoundation.org/networking/netem}}, or rally, which uses the benchmarking tool of OpenStack to test different scenarios given in the parameters.

We use two different network interface to transfer metrics data to a special node that collect all of it and another to link the nodes hosting an OpenStack instance. This is to make sure the collection of data will not be impaired by the delay we apply during the experiments.

We then developed automated python experiments, calling Juice API to implement our test plan presented in section \ref{sec:testplan}, page \pageref{sec:testplan}.

\subsection{OpenStack deployment}

To fully deploy an OpenStack, different approaches coexist, such as Kolla, which provides containers and tools to operate OpenStack infrastructures for production, or DevStack, which is a series of scripts to bring up an Openstack environment. DevStack uses the repositories for OpenStack services and it is then really easy to change the link to whatever fork is needed. This is why we chose this approach because it would be easier to correct any services we want if need be, which would be longer in a container approach.

The deployment of OpenStack relies on DevStack stable/pike and uses default parameters for Keystone (e.g., SQL backend, fernet token, \dots). As said previously, OpenStack, and therefore Devstack, do not natively support CockroachDB. DevStack is then tweaked to ensure Keystone connects to the right DBMS container. The Discovery initiative made a blog article about the support of CockroachDB in Keystone\footnote{\url{https://beyondtheclouds.github.io/blog/openstack/cockroachdb/2017/12/22/a-poc-of-openstack-keystone-over-cockroachdb.html}}.

Juice default configuration for DevStack\footnote{\url{https://github.com/beyondtheclouds/juice/blob/master/ansible/roles/openstack/templates/local.conf.j2}} disables all services except Keystone and its database, but one could test any other service she wants, provided she knows how to use DevStack configuration, though.


\subsection{OpenStack benchmarking}
\label{subsec:benchmarking}

We used mainly OpenStack Rally\footnote{\url{https://github.com/openstack/rally}} to benchmark the performance of Keystone. It generates a load to evaluate how Keystone control plane behaves when scaling.
One Rally executes is load on one instance of OpenStack. This means we can either launch one Rally on the entire cluster or one Rally per instance, which creates a huge burst. Indeed, there are two variables to configure the execution of a Rally scenario: the number of times a task is executed during the scenario and the concurrency, defining the number of parallel task executions. Thus, a scenario with a times of 100 runs one hundred iterations of the scenario by a constant load on the OpenStack instance. A concurrency of 10 specifies that the one hundred iterations are achieved by ten users in a concurrent manner.

This means that when we are executing a burst on 45 OpenStack instances (and so 45 Rallys) with a concurrency of 10 and a times of 100 charges 450 constant transactions on the RDBMS up until the 4,500 iteration are done.

In addition to Rally, Juice can also use sysbench\footnote{\url{https://github.com/akopytov/sysbench}} to benchmark the database more precisely.
TODO


\subsection{Globabl backup}

The execution of a Rally scenario produces a json file. The json file contains a list of entries: one for each iteration of the scenario. An entry then retains the time it takes to complete all Keystone operations involved in the Rally scenario.

Juice \verb+backup+ produces an archive with the corresponding json files to the scenarios ran for each Rally instances as well as general metrics collected over the experiments such as the CPU/RAM consumption, network communications (all stored in a influxdb archive).

If one needs to also get a backup of the studied database, she only needs to add the ansible rule in the backup file of the corresponding database (\url{https://github.com/BeyondTheClouds/juice/blob/master/ansible/roles/database/RDBMS-NAME/tasks/backup.yml}).


\section{Test plan}
\label{sec:testplan}
\subsection{The testbed: Grid'5000}
Grid’5000\footnote{\url{https://www.grid5000.fr/mediawiki/index.php/Grid5000:Home}} is a large-scale and versatile testbed for experiment-driven research in all areas of computer science, with a focus on parallel and distributed computing including Cloud, HPC and Big Data. The platform gives an access to approximately 1000 machines grouped in 30 clusters geographically distributed in 8 sites. This study uses the ecotype cluster made of 48 nodes with each:

\begin{description}
    \item[CPU:] Intel Xeon E5-2630L v4 Broadwell 1.80GHz (2 CPUs/node, 10 cores/CPU)
    \item[Memory:] 128 GB
    \item[Network:]      \begin{itemize}
        \item eth0/eno1, Ethernet, configured rate: 10 Gbps, model: Intel 82599ES 10-Gigabit SFI/SFP+ Network Connection, driver: ixgbe
        \item eth1/eno2, Ethernet, configured rate: 10 Gbps, model: Intel 82599ES 10-Gigabit SFI/SFP+ Network Connection, driver: ixgbe
      \end{itemize}
\end{description}

\subsection{Rally scenarios}

Here is the complete list of rally scenarios considered in this study. Values inside the parentheses refer to the percentage of reads versus the percent of writes on the RDBMS. More information about each scenario is available in the appendix (see, Detailed Rally scenarios, page \pageref{sec:detail-rally}).

\begin{description}
    \item[keystone/authenticate-user-and-validate-token (96.46, 3.54):] Authenticate and validate a keystone token.
    \item[keystone/create-add-and-list-user-roles (96.22, 3.78):] Create user role, add it and list user roles for given user.
    \item[keystone/create-and-list-tenants (92.12, 7.88):] Create a keystone tenant with random name and list all tenants.
    \item[keystone/get-entities (91.9, 8.1):] Get instance of a tenant, user, role and service by id’s. An ephemeral tenant, user, and role are each created. By default, fetches the ’keystone’ service.
    \item[keystone/create-user-update-password (89.79, 10.21):] Create user and update password for that user.
    \item[keystone/create-user-set-enabled-and-delete (91.07, 8.93):] Create a keystone user, enable or disable it, and delete it.
    \item[keystone/create-and-list-users (92.05, 7.95):] Create a keystone user with random name and list all users.
\end{description}



\subsection{Multiple OpenStack instances deployment with the databases}


\subsubsection{MariaDB and Keystone: a single MariaDB}




Figure \ref{fig:key-dep-centralized} depicts the deployment for MariaDB. MariaDB is a centralized RDBMS and thus, the Keystone backend is centralized in the first OpenStack instance.

\begin{wrapfigure}{r}{0.5\textwidth}
  \vspace{-20pt}
  \begin{center}
    \includegraphics[width=0.5\textwidth]{no_push/keystone-centralized.png}
  \end{center}
  \vspace{-20pt}
  \caption{Keystone deployment with a centralized MariaDB.}
  \vspace{-15pt}
  \label{fig:key-dep-centralized}
\end{wrapfigure}

 Other Keystones from the other OpenStack nodes refers to the backend of the first instance. The red logo is an OpenStack instance, the turtle is the Keystone logo.

This kind of deployment is easy to set up since you only need to have the IP address of the node storing MariaDB to add any other node, so the nodes configurations are all the same. MariaDB also has the advantage of having a lot of work done on the performance under different types of workloads.

But this kind of deployment comes with -at least- two possible limitations. First, a centralized RDBMS is a single point of failure that makes all OpenStack instances unusable if it crashes. Second, a network disconnection of, for example, one of the ``bottom'' OpenStack instance with the one that stores MariaDB makes the separated node unusable.



\subsubsection{Galera in multi-master replication mode and Keystone: over replicated Galera}
The figure \ref{fig:key-dep-galera} depicts the deployment for Galera. Galera synchronizes multiple MariaDB databases in an active/active (multi-master) fashion, as previously explained in section \ref{sec:galera-cluster}.

\begin{wrapfigure}{l}{0.5\textwidth}
  \vspace{-20pt}
  \begin{center}
    \includegraphics[width=0.5\textwidth]{no_push/keystone-galera.png}
  \end{center}
  \vspace{-20pt}
  \caption{Keystone deployment with synchronized MariaDB instances thanks to Galera.}
  \vspace{-12pt}
  \label{fig:key-dep-galera}
\end{wrapfigure}

 Thus Keystone’s backend of every OpenStack instance is replicated between all nodes, which allows reads and writes on any instances. The figure semantic is the same as in figure \ref{fig:key-dep-centralized}, but with the added synchronization between the replicated MariaDB thanks to Galera (blue ring).

We use a multi-master topology, which means reads and writes can be done on any nodes. For the State Snapshot Transfer, we use rsync because it is faster and we do not need to reboot the nodes. We also use the \verb+wsrep_sync_wait+ at the maximum level to ensure causality checks on \verb+READ+, \verb+UPDATE+, \verb+DELETE+, \verb+INSERT+, \verb+REPLACE+ and \verb+SHOW+ operations. This means there is a check to see if the node is fully synchronized before these operations are executed.

Regarding possible limitations, few rumors\footnote{\url{http://www.fromdual.com/limitations-of-galera-cluster}} stick to Galera. First, synchronization may suffer from high latency networks. Second, contention during writes on the database may limit its scalability.




\subsubsection{CockroachDB and Keystone: over distributed CockroachDB}
Figure \ref{fig:key-dep-crdb} depicts the deployment for CockroachDB. In this deployment, each OpenStack instance has its own Keystone.

\begin{wrapfigure}{r}{0.5\textwidth}
  \vspace{-20pt}
  \begin{center}
    \includegraphics[width=0.5\textwidth]{no_push/keystone-cockroachdb.png}
  \end{center}
  \vspace{-20pt}
  \caption{Keystone deployment with CockroachDB.}
  \vspace{-12pt}
  \label{fig:key-dep-crdb}
\end{wrapfigure}

The backend is distributed through key-value stores on every OpenStack instance. Meaning, the data a Keystone is sought for is not necessarily in its local key-value store.

CockroachDB is relatively new and we know a few about its limitations, but first, CockroachDB may suffer from high network latency even during reads if the data is located on another node. Second, as Galera, transaction contention may dramatically slow down the overall execution. However, CockroachDB offers locality option (replicas zone) to drive the selection of key-value stores during writes and replication. Thanks to this option it is be possible to mitigate the impact of latency by ensuring that writes happen close the OpenStack operator. Indeed, we can choose to refer to three of the instances as the same region (with \verb+--locality=region=REGION-NAME,datacenter=NODE-NAME+) and then specify that the Keystone table should be only on this region. For example, to specify one replica per region:
\begin{verbatim}
constraints: {
"+region=REGION-NAME-1": 1,
"+region=REGION-NAME-2": 1,
"+region=REGION-NAME-3": 1
}
\end{verbatim}

\subsection{Parameters variations}
\label{subsec:param}
\subsubsection{Number of OpenStack instances}

The OpenStack size defines the number of OpenStack instances deployed for an experiment. It varies between 3, 9 and 45. A value of 3, means Juice deploys OpenStack on three different nodes, 9 on nine different nodes, \dots The value of 45 comes from the maximum number of nodes available on the ecotype Grid’5000 cluster, but Juice is not limited itself.


Experiments that test the impact of the number of OpenStack instances consider a LAN link between each OpenStack instances.

\subsubsection{Delay}

The delay defines the network latency between two OpenStack instances. It is expressed in terms of the time for a signal sent from one node to reach another, (i.e., a value of 50 stands for 100 ms of Round-Trip Time, 150 is 300 ms of RTT). The 0 value stands for LAN speed which is approximately 0.08 ms of RTT on the ecotype Grid’5000 cluster (10 Gbps card).

Juice applies theses network latencies with tc netem. Note that, as previously said, juice applies tc rules on network interfaces that are dedicated to the RDBMS communications. Thus, metrics collection and other network traffics are not limited.

Experiments that test the impact of the network latency are done with 9 OpenStack instances. They make the delay varies by applying traffic shaping homogeneously between the 9 OpenStack instances.

\subsection{How we make the analysis}



\chapter{How do they actually compare}
\label{chap:results}
\section{Number of OpenStack Instances Impact}

This test evaluates how the completion time of Rally Keystone’s scenarios varies, depending on the RDBMS and the number of OpenStack instances. It measures the capacity of a RDBMS to handle lots of connections and requests. In this test, the number of OpenStack instances varies between 3, 9 and 45 and a LAN link inter-connects instances. As explained before, the deployment of the database depends on the RDBMS. With MariaDB, one instance of OpenStack contains the database, and others connect to that one. For Galera and CockroachDB, each OpenStack contains an instance of the RDBMS.

For these experiments, Juice deployed the RDBMS along with OpenStack instances and plays Rally scenarios listed in section \ref{rally-scenarios} page \pageref{rally-scenarios}. Juice runs Rally scenarios in both single and heavy modes. Results are presented in two next sections. The Juice implementation for these experiments is available on GitHub at \url{https://github.com/BeyondTheClouds/juice/tree/master/experiments}, except for the locality, which is available at \url{https://github.com/Marie-Donnie/juice/blob/regions/experiments.py}. This experiments defines the different groups and the delays to apply on the network, inside and between the groups.

We then filter experiments to only keep those when delay is 0 in a single Rally mode. Groups results by scenario’s name, RDBMS technology and number of OpenStack instances. Then, filters results above the 95th quantile. In the plot, the $\lambda$ Greek letter stands for the failure rate and $\sigma$ for the standard deviation.

\subsection{Light load}

Figure \ref{fig:oss-impact-light} presents the mean completion time (in seconds) of Keystone scenarios in a light Rally mode. In the figure, columns presents results of a specific scenario: the first column presents results for Authenticate User and Validate Token, the second for Create Add and List User Role, and so forth. Rows present results with a specific RDBMS: the first row presents results for MariaDB, second for Galera and third for CockroachDB. The figure presents results with stacked bar charts. Each bar presents the result for a specific number of OpenStack instances (i.e., 3, 9 and 45) and stacks completion times of each Keystone operations inside a scenario.

Cdf\ref{fig:oss-impact-light-cdf}


\begin{figure}[H]
  \vspace{-10pt}
  \centering
  \centerline{\includegraphics[width=1\textwidth]{no_push/short-oss-impact-light.png}}
  \vspace{-5pt}
  \caption{Impact of the Number of OpenStack Instances on the Completion Time under Light Load – Median Time for Each Operations.}
  \vspace{-5pt}
  \label{fig:short-oss-impact-light}
\end{figure}

\subsection{Heavy load}

\begin{figure}[H]
  \vspace{-10pt}
  \centering
  \centerline{\includegraphics[width=1\textwidth]{no_push/short-oss-impact-high.png}}
  \vspace{-5pt}
  \caption{Impact of the Number of OpenStack Instances on the Completion Time under Heavy Load – Median Time for Each Operations.}
  \vspace{-5pt}
  \label{fig:short-oss-impact-high}
\end{figure}


\section{Delay Impact}

In this test, the size of the database cluster is 9 and the delay varies between LAN, 100 and 300 ms of RTT. The test evaluates how the completion time of Rally scenarios varies, depending of RTT between nodes of the swarm.



\subsection{Light load}

\begin{figure}[H]
  \vspace{-10pt}
  \centering
  \centerline{\includegraphics[width=1\textwidth]{no_push/short-delay-impact-light.png}}
  \vspace{-5pt}
  \caption{Impact of the Delay on the Completion Time under Light Load – Median Time for Each Operations.}
  \vspace{-5pt}
  \label{fig:short-delay-impact-light}
\end{figure}

Figure \ref{fig:delay-impact-light}

cdf\ref{fig:delay-impact-light-cdf}

\subsection{Heavy load}

\begin{figure}[H]
  \vspace{-10pt}
  \centering
  \centerline{\includegraphics[width=1\textwidth]{no_push/short-delay-impact-high.png}}
  \vspace{-5pt}
  \caption{Impact of the Delay on the Completion Time under Heavy Load – Median Time for Each Operations.}
  \vspace{-5pt}
  \label{fig:short-delay-impact-high}
\end{figure}

Figure \ref{fig:delay-impact-high}

cdf \ref{fig:delay-impact-high-cdf}

\section{Locality Impact}

This study considers two kinds of OpenStack instances deployments. This first one, called uniform, defines a uniform distribution of the network latency between OpenStack instances. For instance, 300 ms of RTT between all the 9 OpenStack instances. The second deployment, called hierarchical, maps to a more realistic view, like in cloud computing, with groups of OpenStack instances connected through a low latency network (e.g., 3 OpenStack instances per group deployed in the same country, and accessible within 20 ms of RTT). And high latency network between groups (e.g. 150 ms of RTT between groups deployed in different countries).

\begin{figure}[H]
  \vspace{-10pt}
  \centering
  \centerline{\includegraphics[width=0.6\textwidth]{no_push/short-zones-impact-light.png}}
  \vspace{-5pt}
  \caption{Impact of the Locality on the Completion Time under Light Load – Median Time for Each Operations.}
  \vspace{-5pt}
  \label{fig:short-zones-impact-light}
\end{figure}


Figure \ref{fig:zones-impact-light}

cdf \ref{fig:zones-impact-light-cdf}


\addcontentsline{toc}{chapter}{\textbf{Conclusion}}
\chapter*{Conclusion}

An ever growing number of applications requires a wide distribution over the globe of computing resourcing, such as the Internet of Things (IoT). To manage this kind of ``Edge Computing'', the Discovery initiative chose to modify the OpenStack platform in order to avoid reinventing the wheel.

Some OpenStack components do not fulfill edge infrastructure needs. Especially, the Relational DataBase Management System is difficult to distribute accross the infrastructure because it has to cope with lots of connections and high latency network, making it crucial to consider the locality of the client.

Keystone, OpenStack authentication service, is using by default MariaDB to store all its data. Moreover, experimentations made on Keystone and their results make sense for other OpenStack services (Nova, Neutron, \dots) which also uses MariaDB by default.

To test the performances of different backends for Keystone and to better have an idea of the behavior of OpenStack in the context of the edge using locality, we made an evaluation of the performances of different RDBMS, using different topologies.

In order to get this evaluation, I developed Juice, a framework to deploy OpenStack using different backends on a varying number of servers, varying network delays, and a simulation of an edge infrastructure, with heterogenous network delays.

The results of this evaluation show that, though it is a single point of failure, a centralized MariaDB works pretty well under an increasing number of nodes and latency. MariaDB Galera Cluster handles pretty well high latencies despite what we first thought, but has serious difficulties to cope with a high number of nodes. Finally, CockroachdB manages pretty well numerous nodes, but can't handle high delays.

What was even more interesting were the results from the usage of the replication zones from CockroachDB. This showed that we can bypass the difficulties of a distributed database by considering the locality of the client. There are still disadvantages about CockroachDB; e.g. the fact that the locality is parametrized generally (for all the database), and we want to be able to configure it per query, e.g. add a user where he will use services (for Keystone), get the state of a booted VM on a server (for Nova). Parametrization per query ensures that data are written where they are supposed to be, closer to the client, which provides a resiliency to disconnections and network partitions.

This means that, though CockroachDB is not entirely what we need, it has proved that considering the locality of the client is a good way to continue the work for the Discovery initiative. This parametrization per query raises new challenges such as where to put the locality information and modify the queries to consider it.


\clearpage
\appendix
%\addcontentsline{toc}{section}{\textbf{Appendix}}
\chapter{Appendix}
\label{sec:appendix}



\section{Detailed Rally scenarios}
\label{sec:detail-rally}

(Extract from the blog article, from the work of Ronan-Alexandre Cherrueau)

The \%reads/\%writes ratio is computed on Mariadb. The gauging code reads values of status variables \verb+Com\_xxx+ that provide statement counts over all connections (with \verb+xxx+ standing for \verb+SELECT+, \verb+DELETE+, \verb+INSERT+, \verb+UPDATE+, \verb+REPLACE+ statements). The SQL query that does this job is available in listing \ref{lst:rw} and returns the total number of reads and writes since the database started. That SQL query is called before and after the execution of one Rally scenario. After and before values are then subtracted to compute the number of reads and writes performed during the scenario and finally, compared to compute the ratio.

\begin{lstlisting}[caption={Total number of reads and writes performed on MariaDB since the last reboot}\label{lst:rw}, language=SQL, captionpos=b]
SELECT
  SUM(IF(variable_name = 'Com_select', variable_value, 0))
     AS `Total reads`,
  SUM(IF(variable_name IN ('Com_delete',
                           'Com_insert',
                           'Com_update',
                           'Com_replace'), variable_value, 0))
     AS `Total writes`
FROM  information_schema.GLOBAL_STATUS;
\end{lstlisting}

Note that \%reads/\%writes may be a little bit more in favor of reads than what it is presented here because the following also takes into account the creation/deletion of the Rally context. A basic Rally context for a Keystone scenario is \verb+{"admin_cleanup@openstack": ["keystone"]}+. We are not sure what does this context do exactly though, maybe it only creates an admin user… This context may be extended by other inserts specified in the scenario definition (under the \verb+context+ key; see scenario definition for keystone/create-add-and-list-user-roles).

The Juice implementation for this gauging is available on GitHub at \url{https://github.com/rcherrueau/juice/blob/02af922a7c3221462d7106dfb2751b3be709a4d5/experiments/read-write-ratio.py}.


\subsection{keystone/authenticate-user-and-validate-token}

Description: authenticate and validate a keystone token.

Definition Code: \url{https://github.com/openstack/rally-openstack/blob/master/samples/tasks/scenarios/keystone/authenticate-user-and-validate-token.yaml}

Source Code: \url{https://github.com/openstack/rally-openstack/blob/master/rally\_openstack/scenarios/keystone/basic.py#L111-L120}

List of keystone functionalities:
\begin{itemize}
\item    keystone\_v3.fetch\_token
\item    keystone\_v3.validate\_token
\end{itemize}
\%Reads/\%Writes: 96.46/3.54

Number of runs: 20

\subsection{keystone/create-add-and-list-user-roles}

Description: create user role, add it and list user roles for given user.

Definition Code: \url{https://github.com/openstack/rally-openstack/blob/master/samples/tasks/scenarios/keystone/create-add-and-list-user-roles.yaml}

Source Code: \url{https://github.com/openstack/rally-openstack/blob/master/rally\_openstack/scenarios/keystone/basic.py#L214-L228}

List of keystone functionalities:
\begin{itemize}
\item    keystone\_v3.create\_role
\item    keystone\_v3.add\_role
\item    keystone\_v3.list\_roles
\end{itemize}
\%Reads/\%Writes: 96.22/3.78

Number of runs: 100

\subsection{keystone/create-and-list-tenants}

Description: create a keystone tenant with random name and list all tenants.

Definition Code: \url{https://github.com/openstack/rally-openstack/blob/master/samples/tasks/scenarios/keystone/create-and-list-tenants.yaml}

Source Code: \url{https://github.com/openstack/rally-openstack/blob/master/rally\_openstack/scenarios/keystone/basic.py#L166-L181}

List of keystone functionalities
\begin{itemize}
\item    keystone\_v3.create\_project
\item    keystone\_v3.list\_projects
\end{itemize}
\%Reads/\%Writes: 92.12/7.88

Number of runs: 10

\subsection{keystone/get-entities}

Description: get instance of a tenant, user, role and service by id’s. An ephemeral tenant, user, and role are each created. By default, fetches the ’keystone’ service.

List of keystone functionalities:
\begin{itemize}
\item    keystone\_v3.create\_project
\item    keystone\_v3.create\_user
\item    keystone\_v3.create\_role
    \begin{itemize}
    \item    keystone\_v3.list\_roles
    \item    keystone\_v3.add\_role
    \end{itemize}
\item    keystone\_v3.get\_project
\item    keystone\_v3.get\_user
\item    keystone\_v3.get\_role
\item    keystone\_v3.list\_services
\item    keystone\_v3.get\_services
\end{itemize}
\%Reads/\%Writes: 91.9/8.1

Definition Code: \url{https://github.com/openstack/rally-openstack/blob/master/samples/tasks/scenarios/keystone/get-entities.yaml}

Source Code: \url{https://github.com/openstack/rally-openstack/blob/master/rally\_openstack/scenarios/keystone/basic.py#L231-L261}

Number of runs: 100


\subsection{keystone/create-user-update-password}

Description: create user and update password for that user.

List of keystone functionalities:
\begin{itemize}
\item    keystone\_v3.create\_user
\item    keystone\_v3.update\_user
\end{itemize}
\%Reads/\%Writes: 89.79/10.21

Definition Code: \url{https://github.com/openstack/rally-openstack/blob/master/samples/tasks/scenarios/keystone/create-user-update-password.yaml}

Source Code: \url{https://github.com/openstack/rally-openstack/blob/master/rally\_openstack/scenarios/keystone/basic.py#L306-L320}

Number of runs: 100


\subsection{keystone/create-user-set-enabled-and-delete}

Description: create a keystone user, enable or disable it, and delete it.

List of keystone functionalities:
\begin{itemize}
\item    keystone\_v3.create\_user
\item    keystone\_v3.update\_user
\item    keystone\_v3.delete\_user
\end{itemize}
\%Reads/\%Writes: 91.07/8.93

Definition Code: \url{https://github.com/openstack/rally-openstack/blob/master/samples/tasks/scenarios/keystone/create-user-set-enabled-and-delete.yaml}

Source Code: \url{https://github.com/openstack/rally-openstack/blob/master/rally\_openstack/scenarios/keystone/basic.py#L75-L91}

Number of runs: 100

\subsection{keystone/create-and-list-users}

Description: create a keystone user with random name and list all users.

List of keystone functionalities:
\begin{itemize}
\item    keystone\_v3.create\_user
\item    keystone\_v3.list\_users
\end{itemize}
\%Reads/\%Writes: 92.05/7.95

Definition Code: \url{https://github.com/openstack/rally-openstack/blob/master/samples/tasks/scenarios/keystone/create-and-list-users.yaml}

Source Code: \url{https://github.com/openstack/rally-openstack/blob/master/rally\_openstack/scenarios/keystone/basic.py#L145-L163}

Number of runs: 100

\section{Results}
\label{results}

\subsection{Number of OpenStack instances impact under light load}

Figures \ref{fig:oss-impact-light}, \ref{fig:oss-impact-light-cdf} and \ref{fig:oss-impact-light-lreg} result from the experiments about the number of OpenStack instances impact under a light load.


\begin{figure}[h]
  %\vspace{-10pt}
  \centering
  \centerline{\includegraphics[width=1.4\textwidth, angle=90]{no_push/oss-impact-light.png}}
  %\vspace{-5pt}
  \caption{Impact of the Number of OpenStack Instances on the Completion Time under Light Load – Median Time for Each Operations.}
  %\vspace{-5pt}
  \label{fig:oss-impact-light}
\end{figure}


\begin{figure}[h]
%  \vspace{-10pt}
  \centering
  \centerline{\includegraphics[width=1.4\textwidth, angle=90]{no_push/oss-impact-light-cdf.png}}
%  \vspace{-5pt}
  \caption{Impact of the Number of OpenStack Instances on the Completion Time under Light Load – Cummulative Distibution}
%  \vspace{-5pt}
  \label{fig:oss-impact-light-cdf}
\end{figure}


\begin{figure}[h]
%  \vspace{-10pt}
  \centering
  \centerline{\includegraphics[width=1.4\textwidth, angle=90]{no_push/oss-impact-light-lreg.png}}
%  \vspace{-5pt}
  \caption{Impact of the Number of OpenStack Instances on the Completion Time under Light Load – Linear Regression}
%  \vspace{-5pt}
  \label{fig:oss-impact-light-lreg}
\end{figure}

\subsection{Number of OpenStack instances impact under heavy load}

Figures \ref{fig:oss-impact-high} and \ref{fig:oss-impact-high-cdf} result from the experiments about the number of OpenStack instances impact under a heavy load.


\begin{figure}[h]
%  \vspace{-10pt}
  \centering
  \centerline{\includegraphics[width=1.4\textwidth, angle=90]{no_push/oss-impact-high.png}}
%  \vspace{-5pt}
  \caption{Impact of the Number of OpenStack Instances on the Completion Time under Heavy Load – Median Time for Each Operations.}
%  \vspace{-5pt}
  \label{fig:oss-impact-high}
\end{figure}


\begin{figure}[h]
%  \vspace{-10pt}
  \centering
  \centerline{\includegraphics[width=1.4\textwidth, angle=90]{no_push/oss-impact-high-cdf.png}}
%  \vspace{-5pt}
  \caption{Impact of the Number of OpenStack Instances on the Completion Time under Heavy Load – Cummulative Distibution}
%  \vspace{-5pt}
  \label{fig:oss-impact-high-cdf}
\end{figure}

\subsection{Delay impact under light load}

Figures \ref{fig:delay-impact-light} and \ref{fig:delay-impact-light-cdf} result from the experiments about the delay impact under a light load.

\begin{figure}[h]
%  \vspace{-10pt}
  \centering
  \centerline{\includegraphics[width=1.4\textwidth, angle=90]{no_push/delay-impact-light.png}}
%  \vspace{-5pt}
  \caption{Impact of the Delay on the Completion Time under Light Load – Median Time for Each Operations.}
%  \vspace{-5pt}
  \label{fig:delay-impact-light}
\end{figure}

\begin{figure}[h]
%  \vspace{-10pt}
  \centering
  \centerline{\includegraphics[width=1.4\textwidth, angle=90]{no_push/delay-impact-light-cdf.png}}
%  \vspace{-5pt}
  \caption{Impact of the Delay on the Completion Time under Light Load – Cummulative Distibution.}
%  \vspace{-5pt}
  \label{fig:delay-impact-light-cdf}
\end{figure}


\subsection{Delay impact under heavy load}
Figures \ref{fig:delay-impact-high} and \ref{fig:delay-impact-high-cdf} result from the experiments about the delay impact under a heavy load.


\begin{figure}[h]
%  \vspace{-10pt}
  \centering
  \centerline{\includegraphics[width=1.4\textwidth, angle=90]{no_push/delay-impact-high.png}}
%  \vspace{-5pt}
  \caption{Impact of the Delay on the Completion Time under Heavy Load – Median Time for Each Operations.}
%  \vspace{-5pt}
  \label{fig:delay-impact-high}
\end{figure}

\begin{figure}[h]
%  \vspace{-10pt}
  \centering
  \centerline{\includegraphics[width=1.4\textwidth, angle=90]{no_push/delay-impact-high-cdf.png}}
%  \vspace{-5pt}
  \caption{Impact of the Delay on the Completion Time under Heavy Load – Cummulative Distibution.}
%  \vspace{-5pt}
  \label{fig:delay-impact-high-cdf}
\end{figure}

\subsection{Locality impact under light load}

Figures \ref{fig:zones-impact-light} and \ref{fig:zones-impact-light-cdf} result from the experiments about the locality impact under a light load.

\begin{figure}[h]
%  \vspace{-10pt}
  \centering
  \centerline{\includegraphics[width=1.4\textwidth, angle=90]{no_push/zones-impact-light.png}}
%  \vspace{-5pt}
  \caption{Impact of the Locality on the Completion Time under Light Load – Median Time for Each Operations.}
%  \vspace{-5pt}
  \label{fig:zones-impact-light}
\end{figure}

\begin{figure}[h]
%  \vspace{-10pt}
  \centering
  \centerline{\includegraphics[width=1.4\textwidth, angle=90]{no_push/zones-impact-light-cdf.png}}
%  \vspace{-5pt}
  \caption{Impact of the Locality on the Completion Time under Light Load – Cummulative Distibution.}
%  \vspace{-5pt}
  \label{fig:zones-impact-light-cdf}
\end{figure}


\clearpage
\phantomsection
\addcontentsline{toc}{chapter}{\textbf{Bibliography}}


\printbibliography
%% \printbibliography[title={Book references},type=book]
%% \printbibliography[title={Article references},type=article]
%% \printbibliography[title={Other references}, nottype=article, nottype=book]



\end{document}
