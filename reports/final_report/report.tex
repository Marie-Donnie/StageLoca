
\documentclass[a4paper, 10pt, titlepage]{report}

%% Mandatory stuff
\usepackage[utf8]{inputenc}
\usepackage[T1]{fontenc}
%% --


%% Color package for colored links
\usepackage[usenames,dvipsnames]{xcolor}
\definecolor{blues}{RGB}{153,50,204}
%% --


%% Code listings
\usepackage{listings}
% UTF8 compatibility options
% UTF8 chars are OK in code comment
% but not elsewhere
\lstset{extendedchars=false, texcl}
% Eye candy

%% For temporary redefining the geometry of titlepage.
\usepackage{geometry}

\lstdefinestyle{python}{
  language=python,
  basicstyle=\footnotesize\rmfamily,
  identifierstyle=,
  commentstyle=\itshape,
  keywordstyle=,
  stringstyle=,
  showspaces=false,
  showstringspaces=false,
  tabsize=3,
  breaklines=true
}

\newcommand{\code}[1]{\lstinline!#1!}

\lstset{style=python}

%% Theorems environments
\usepackage{amsmath}
\usepackage{amsthm}

%% Power to the tikz!
\usepackage{tikz}
%\usepackage{graphicx}
%% --

%% Page layout
\usepackage{makeidx}
\makeindex

%% Plain page headings
\pagestyle{headings}

%% Smaller space between heading and matter
\setlength\headsep{0.5in}
 %% --

%% Larger page
%%\usepackage[left=4cm,right=4cm,top=5cm,bottom=4cm]{geometry}

%% Adjust toc depth
\setcounter{tocdepth}{4}

%% Dotted toc
\usepackage{tocloft}
\renewcommand{\cftsecleader}{\cftdotfill{\cftdotsep}}

%% Figures where I want them
\usepackage{float}

%% Nice caption
\usepackage[font={footnotesize}]{caption}

%% Footnote where they belong
\usepackage[bottom]{footmisc}

%% Comments
\newcommand{\comment}[1]{}


%% Removes the ``chapter'' heading without the chapter number
%% \usepackage{titlesec}
%% \titleformat{\chapter}[display]
%%   {\normalfont\bfseries}{}{0pt}{\Huge \thechapter\ - }

%% Image in title
\usepackage{graphicx}

%% Images wrapped in text
\usepackage{wrapfig}

%% Subfigures
\usepackage{subcaption}

%% Bibliography
\usepackage[style=numeric]{biblatex}

\addbibresource{biblio.bib}

%% Hyperlinks for references, in color-O-vision and nice for long urls
%% MUST BE THE LAST PACKAGE
\PassOptionsToPackage{hyphens}{url}
\usepackage[pdftex,%
  pdfauthor={Marie Delavergne},%
  pdftitle={Report for  Master 2 ALMA},%
  pdfsubject={A middleware for a NewSQL database that reflects the client application locality},%
  colorlinks,%
  linkcolor=blues,%
  urlcolor=blues,%
  plainpages=false]{hyperref}
%% --


\begin{document}
%
%% \maketitle
\newgeometry{bottom=2.5cm}

\begin{titlepage}
  \centering
      {\scshape\LARGE\bfseries Considering the client locality in the context of Edge Computing \par}
      \vspace{1cm}
             {\Large - Internship Report for Master 2 ALMA -\par}
             {\Large January--June 2018 \par}
             \vspace{1.5cm}
               	    {\Large\itshape Marie Delavergne\par}

	            \vfill

                    % Bottom of the page

                    {\large Supervisors: Ronan-Alexandre Cherrueau (Inria)\\
                      \hspace{3.45cm}Achour Mostefaoui (University of Nantes)\par}
                          \vspace{2cm}
                    {\large Stack - DAPI - LS2N - Inria - Discovery Initiative\par}
	            {\large University of Nantes\par}
                          \vspace{0.5cm}
                    \begin{figure}[!h]\centering
                      \begin {minipage}{0.3\textwidth}                        \centerline{\includegraphics[width=\linewidth]{no_push/inr_logo_eng_rouge.png}}
                        \label{Fig:inria}
                      \end{minipage}
                      \begin{minipage}{0.3\textwidth}                   \centerline{\includegraphics[width=\linewidth]{no_push/logo_un2012quadri_larg40.png}}
                        \label{Fig:univnantes}
                      \end{minipage}


                    \end{figure}


\end{titlepage}
\restoregeometry

\clearpage
\tableofcontents
\newpage

\section*{Aknowledgement}

I would like to thank first Ronan-Alexandre for his supervision, knowledge, work, help, support and patience. The Discovery and Stack team for welcoming me and helping me. Thanks to Adrien and Inria for trusting me and sending me to Vancouver. To Achour for his teachings about the consensuses and logical time. To Olivier Grasset and the ESA for the inspiration for the name of my framework. And of course Florent for his help and support.

\newpage

\section*{Abstract}

An ever growing number of applications requires a wide distribution over the globe of computing resources, such as the Internet of Things (IoT). To answer this need, we require Edge infrastructures fully automated, using a resource management system able to provide everything needed to handle the difficulties. These are managing hundreds of datacenters with dozen of servers, heterogeneous wide networks with possible disconnections and partitions. Building this kind of systems from scratch is too much of a challenge, thus we think it is better to change an existing, widely used IaaS manager in order to satisfy the requirements.
OpenStack is made to be scalable, except for some components such as the centralized database, which represents a single point of failure. NewSQL databases, such as CockroachDB, are made to scale under OLTP while complying the relational model which OpenStack uses. While an Edge infrastructure manages high latency connections, we think that is really important to consider the locality of the client to reduce network latency.
However, we do not know how would OpenStack behave in an Edge context, or the concrete impact of considering the locality.
We present here an evaluation of the performances of Keystone, an OpenStack service, over three RDBMS, namely MariaDB, MariaDB Galera Cluster and CockroachDB. Galera is the clustering solution used by MariaDB.
We found that MariaDB Galera Cluster handles pretty well the high delays but not the number of nodes, and conversely for CockroachDB. But more important, we confirmed that CockroachDB has excellent performances when we consider locality.
We think that there is still more work to do regarding the parametrization of the locality to refine it up to the query.


\section*{Résumé}

De plus en plus d'applications utilisent une distribution globale de ressources de calcul, telles que l'Internet des Objets. Pour répondre à ce besoin, il est nécessaire que les infrastructures d'\emph{informatique en périphérie} soient entièrement automatisées en utilisant un système de gestion de ressources capable de remplir tous les besoins de ces infrastructures. Elles nécessitent de faire face à plusieurs difficultés comme gérer des centaines de centres de données comportant eux-mêmes des dizaines de serveurs, gérer le fait d'avoir des réseaux hétérogènes avec de possibles déconnexions et des partitions. Construire un tel système de zéro est bien trop difficile et nous pensons alors qu'il est plus intelligent de modifier un gestionaire d'Infrastructure-en-tant-que-Service pour satisfaire ces besoins.
OpenStack a été fait pour pouvoir passer à l'échelle, sauf certains composants tels que la base de données centralisée, qui représente un point unique de défaillance. Les bases de données NewSQL telles que CockroachDB ont été faites pour gérer du traitement transactionnel en ligne tout en fournissant une interface satisfaisant le modèle relationnel qu'utilise OpenStack. Bien qu'une infrastructure en périphérie doivent gérer des connections à forte latences, nous pensons qu'il est nécessaire de prendre en compte la localité du client pour réduire les latences lors des requêtes.
Cependant, nous ignorons comment OpenStack se comporterait réellement dans un contexte d'informatique en périphérie, ou l'impact concret de cette prise en compte de la localité.
Nous présentons ici une évaluation des performances de Keystone, un service d'OpenStack, sur trois systèmes de gestion de base de données relationnelles : MariaDB, MariaDB Galera Cluster et CockroachDB. Galera est la solution de grappe de serveurs utilisée par MariaDB.
Nos résultats montrent que MariaDB Galera Cluster gère correctement des fortes latences réseaux, mais pas le nombre de serveurs, et inversement pour CockroachDB. Mais plus important, nous avons confirmé que CockroachDB obtient d'excellentes performances lorsque nous prenons en compte la localité.
Nous pensons qu'il y a toujours du travail à effectuer concernant la paramétrisation de la localité pour qu'elle soit aussi fine qu'au niveau d'une requête.


\addcontentsline{toc}{chapter}{\textbf{Introduction}}
\chapter*{Introduction}

Consider a bank. In the past, every transaction was recorded on paper, everything was computed manually and people working in the bank probably made mistakes. Nowadays, everything is digital and the bank software can not be fallible. The software must also store all the information about its clients, accounts, transactions, \dots.

In the digital word, this information is stored in databases, which are exactly organized collections of data. The databases can store all sorts of data, like words, numbers, or even texts. To store these data, they traditionally use different tables for different information, linked and identified by an unique identifier (ID). As we can see on tables \ref{tab:clientname} and \ref{tab:accountbalance}, the ID is the account's number and it is used to point a client name to her account balance. A single row refers to the same ID. In this example, Scrooge McDuck, with the account number 1, has a balance of \$300,000,000,000,000. Donald Duck, on the other hand, only has \$5 on his account numbered 2.
\begin{table}[H]
  \begin{minipage}[b]{0.45\linewidth}\centering
\begin{tabular}{|c|c|c|}
\hline
ID & First Name & Last Name \\
\hline
1 & Scrooge & McDuck \\
2 & Donald & Duck \\
\hline
\end{tabular}
\caption{Client name table}
\label{tab:clientname}
  \end{minipage}
\begin{minipage}[b]{0.45\linewidth}\centering

\begin{tabular}{|c|c|}
\hline
ID & Balance \\
\hline
1 & 300,000,000,000,000 \\
2 & 5 \\
\hline
\end{tabular}
\caption{Account balance table}
\label{tab:accountbalance}
\end{minipage}
\end{table}


My internship has been about considering the client locality in a NewSQL database. This kind of database scales while keeping the properties required to enforce proper transactions. Its entries are distributed across a cluster of nodes. What we want is to prioritize the access to one node rather the entirety of the cluster of nodes. This report will explain how we evaluated performances of different topologies and databases to pave the way towards the locality we want.

"We" designates the Discovery initiative\cite{discovery}. This is a Inria Project Lab (IPL) that involves the Stack, Avalon, Corse, Myriads and Resist teams, along with Orange and some ponctual help from Renater or RedHat. The initiative aims at implementing a fully decentralized IaaS manager, by using a revised OpenStack to make it cooperative. It studies the possiblity of infrastructures for the massively distributed cloud computing in small edge datacenters. Those infrastructures aim at avoiding the problems of datacenter scaling and limiting the latency by bringing the compute nodes closer to the users.


The Stack team, in which I worked is part of the LS2N, a ``Unité Mixte de Recherche'' (Mixed research unit) which involves the University of Nantes, Centrale Nantes, IMT Atlantique, CNRS and Inria. I personnaly worked for five months for Inria at IMT Atlantique. Inria is the French Institute for Research in Computer Science and Automation, a national research institution for computer science and applied mathematics. The IMT Atlantique Bretagne Pays de la Loire is a engineering school where most of the Stack team works.

Some work has previously been done by the Discovery initiative towards getting the compute closer to the users, and some was also ongoing in parallel. My contribution was to make Juice\footnote{\url{https://github.com/BeyondTheClouds/juice}}, a framework to deploy and evaluate OpenStack performances on different databases and topologies. It lead to a presentation at the OpenStack Summit in Vancouver\footnote{\url{https://www.openstack.org/videos/vancouver-2018/keystone-in-the-context-of-fogedge-massively-distributed-clouds}} and a blog article detailing the experiments and results\footnote{\url{https://beyondtheclouds.github.io/blog/openstack/cockroachdb/2018/06/04/evaluation-of-openstack-multi-region-keystone-deployments.html}}. The main goal of my work was to compare different kind of database under different topologies to confirm that a locality-based architecture was a good way to go.

To better explain the work done in this internship, I will first detail the context, then we will see what already exists and what solutions we considered, then I will describe the experiment plans and analyze the details to finally conclude on the future work to be done.



\chapter{Openstack at the Edge}
\label{chap:os-edge}

\subsection{Edge computing}


\subsection{Problem}



\chapter{Relational database management systems considered}
\label{chap:rdmbs-considered}
Two databases were considered: one distributed, CockroachDB, and the other replicated, MariaDB Galera Cluster. We also used MariaDB as a control for the other databases, keeping in mind that it is a single point of failure, so we can not really use it in our context of edge computing.

\subsection{MariaDB Galera Cluster}

MariaDB, an open-source fork of MySQL, uses Galera Cluster as a synchronous multi-master cluster. It means that all nodes in the cluster are masters, with an active-active synchronous replication, so it is possible to read or write on every node at any time.

\begin{figure}[H]
  \vspace{-10pt}
  \centering
  \centerline{\includegraphics[width=1\textwidth]{no_push/galera-2.pdf}}
  \vspace{-5pt}
  \caption{MariaDB Galera Cluster}
  \vspace{-5pt}
  \label{fig:MGC}
\end{figure}

To say it in a more understandable way, MariaDB Galera Cluster allows to have the exact same database on every nodes thanks to a synchronous replication.

To dive more into details, each time a transaction is requested by a client on a node, it is processed as usual until the client issues a commit. The process is stopped and all changes made to the database in the transaction are collected in a ``write-set'', along with the primary keys of the changed rows. The write-set is then broadcasted to every nodes. It then undergoes a deterministic certification test which uses the given primary keys. It actually checks all transactions between the current transaction and the last successful one to determine whether the primary keys involved conflicts between each others.

If the check fails, the whole transaction is rollbacked, and if it succeeds it is applied on all nodes then committed.

\begin{figure}[H]
  \vspace{-10pt}
  \centering
  \centerline{\includegraphics[width=0.85\textwidth]{no_push/certification_based_replication.png}}
  \vspace{-5pt}
  \caption{Certification Based Replication from \url{http://galeracluster.com} }
  \vspace{-5pt}
  \label{fig:certificationcommit}
\end{figure}


This is pretty efficient since it only needs a broadcast to make the replication, which means. But this means that when it fails, the entire transaction must be retried and so it may lead to more conflicts and even deadlocks.

It also have advantages like granting there will be no data loss when nodes crash, providing a high availability across the cluster.

Galera Cluster uses a virtual synchrony that unifies the data delivery, providing a formalism for its semantics. Since this does not guarantee temporal synchrony, it also implements a flow control to keep nodes synchronized.

The multi-master replication allows only read-repeatable isolation at best for transactions, but more will be said on the subject in section \ref{sec:serial}, \pageref{sec:serial}.


\subsection{CockroachDB}
CockroachDB is a NewSQL database that uses the Raft protocol (an alternative version to Lamport's Paxos consensus protocol).
It uses the SQL API to enable SQL requests on every nodes. These requests are translated to key-value operations and -if needed- distributed across the cluster.

CockroachDB implements a single, monolithic sorted map for the keys and values stored, as seen in \ref{fig:cockroachdb}. This map is divided in ranges, which are continuous chunks of this map, with every key being in a single range, so the ranges will not overlap. Each ranges are then replicated (three replicas per range) and finally distributed across the cluster nodes.\cite{CRDB:automatedoperations}

\begin{figure}[H]
  \vspace{-10pt}
  \centering
  \centerline{\includegraphics[width=1\textwidth]{no_push/cockroachdb.png}}
  \vspace{-5pt}
  \caption{CockroachDB ranges, replicas and leaseholders}
  \vspace{-5pt}
  \label{fig:cockroachdb}
\end{figure}

One of the replicas acts as the leaseholder, a sort of leader that coordinates all reads and writes for the range. A read only requires the leaseholder.
When a write is requested, the leaseholder prepares to append it to its log, forward the request to the replicas and when the quorum is achieved, commit the change by add it in the log. The quorum is an agreement from two out of the three replicas to make the change.

\begin{figure}[H]
  \vspace{-10pt}
  \centering
  \centerline{\includegraphics[width=0.85\textwidth]{no_push/commit-cockroach.png}}
  \vspace{-5pt}
  \caption{CockroachDB commit: only a quorum of 2 is required to commit }
  \vspace{-5pt}
  \label{fig:cockroachdb-commit}
\end{figure}

To implement an SQL API, Cockroach uses an encoding tool to go from SQL data to a key-value store\cite{CRDB:mapKV}. As an example, a bit of code:
\begin{verbatim}
CREATE TABLE test (
      key       INT PRIMARY KEY,
      floatVal  FLOAT,
      stringVal STRING
)

INSERT INTO test VALUES (10, 4.5, "hello”)
\end{verbatim}

This row would be stored as:

\begin{center}
\begin{tabular}{|l|l|}
\hline
Key & Value\\
\hline
/test/10/floatVal & 4.5\\
/test/10/stringVal & "hello"\\
\hline
\end{tabular}
\end{center}

Here, \verb~/test/~ is a placeholder for the table ID and the \verb~/*Val~ are placeholders for the column ID used in CockroachDB. Each non-primary key column are stored under a separate key that is prefixed by the primary key (following the table ID) and suffixed by the column ID.

\subsubsection{Raft}

\cite{DBLP:conf/usenix/OngaroO14}

\label{sec:serial}
\subsection{Serializability}
Difference between Serializable and Snapshot Isolations from \href{https://www.cockroachlabs.com/blog/what-write-skew-looks-like/}{Cockroach Labs}:


Consider the following transactions:
\begin{table}[H]
  \centering
  \begin{tabular}{ c }
    $P:   y \gets x$\\
    $Q:   x \gets y$\\
  \end{tabular}
\end{table}


There are two possible serial executions: $P$ then $Q$ OR $Q$ then $P$
\begin{table}[H]
  \centering
  \begin{tabular}{ r | l }
    (P) $y \gets x$ & $x \gets y$ (Q) \\
    (Q) $x \gets y$ & $y \gets x$ (P) \\
    $x = y$ & $x = y$ \\
  \end{tabular}
\end{table}

Snapshot isolation allows another execution:

\begin{table}[H]
  \centering
  \begin{tabular}{ r | l }
    1. P reads $x$ as $x_1$ & 2. Q reads $y$ as $y_1$ \\
    3. P writes $x_1$ to $y$ & 3. Q writes $y_1$ to $x$ \\
  \end{tabular}
  \\
  $x \neq y$, values have been swapped
\end{table}

Indeed, each transaction kept a consistent view of the database and the writes were not on the same value, so no overlaping of the transaction write set occured, then no retries were triggered.



\chapter{How to compare relational database management systems}
\label{chap:experiments}
what we've done



\chapter{How do they actually compare}
\label{chap:results}
This section present the results and their analysis. To avoid lengths of graphics in this report, only a short version of the results are presented in this section. The entirety of the results, median time for each operations and Cumulative Distribution Functions (CDF) for each scenarios are in the appendix. All results are filtered above the 95th quantile. In the median completion time plot, the $\lambda$ Greek letter stands for the failure rate and $\sigma$ for the standard deviation. For the CDF, the abscissa is the time and the ordinate is the pourcentage of operations made during that time.



The short results are the median time of completion for each operations. Each frame represents a scenarios played on a RDBMS, the abscissa is for the variable parameter, and the ordinate is the completion time of each operation. This completion of the operations is shown thanks to stacked bars. The scenario is Authenticate and Validate Keystone Token (\%reads: 96.46, \%writes: 3.54), except for the locality, where we used Create User and Update its Password (\%reads: 89.79, \%writes: 10.21) since it has enough writings to be interesting regarding the quorum (any reading only need any replicas).




\section{Number of OpenStack Instances Impact}

This test evaluates how the completion time of Rally Keystone’s scenarios varies, depending on the RDBMS and the number of OpenStack instances. It measures the capacity of a RDBMS to handle lots of connections and requests. In this test, the number of OpenStack instances varies between 3, 9 and 45 and a LAN link inter-connects instances. As explained before, the deployment of the database depends on the RDBMS. With MariaDB, one instance of OpenStack contains the database, and others connect to that one. For Galera and CockroachDB, each OpenStack contains an instance of the RDBMS.

For these experiments, Juice deployed the RDBMS along with OpenStack instances and plays Rally scenarios listed in section \ref{rally-scenarios} page \pageref{rally-scenarios}. Juice runs Rally scenarios in both single and heavy modes. Results are presented in the two next subections. The Juice implementation for these experiments is available on GitHub, as explained in the previous section.


\subsection{Light load}

Figure \ref{fig:oss-impact-light} (page \pageref{fig:oss-impact-light}) presents the mean completion time (in seconds) of Keystone scenarios in a light Rally mode. In the figure, columns presents results of a specific scenario: the first column presents results for Authenticate User and Validate Token, the second for Create Add and List User Role, and so forth. Rows present results with a specific RDBMS: the first row presents results for MariaDB, second for Galera and third for CockroachDB. The figure presents results with stacked bar charts. Each bar presents the result for a specific number of OpenStack instances (i.e., 3, 9 and 45) and stacks completion times of each Keystone operations inside a scenario.


Figure \ref{fig:short-oss-impact-light} is the short summary of the results. We can see that the trend is similar for the 3 RDMBS, to the advantage of the centralized MariaDB, followed by MariaDB Galera Cluster and then CockroachDB.
\begin{figure}[H]
  \vspace{-10pt}
  \centering
  \centerline{\includegraphics[width=1\textwidth]{no_push/short-oss-impact-light.png}}
  \vspace{-5pt}
  \caption{Impact of the Number of OpenStack Instances on the Completion Time under Light Load – Median Time for Each Operations – Authenticate and Validate Keystone Token Scenario.}
  \vspace{-5pt}
  \label{fig:short-oss-impact-light}
\end{figure}

We can see in the longer results that MariaDB Galera Cluster begins to have difficulties to handle 45 nodes, to the point were it can not handle the loads of Create User Set Enabled and Delete and Create and List Users. This kind of failure were not really explained and we have had the time to analyze this further.

The corresponding Cumulative Distribution Function (CDF) can be found in figure \ref{fig:oss-impact-light-cdf}, page \pageref{fig:oss-impact-light-cdf}. We can see that for both MariaDB solutions, all the operations the three last scenarios take globally the same time, whereas for CockroachDB, a few operations take a lot of time.

I have added in the appendix the linear regression for these scenarios in Figure \ref{fig:oss-impact-light-lreg} to show that the results are not always perfect (the distribution is sometimes off), and to see how does the slope looks like with such results. Globally, MariaDB Galera Cluster's slopes are stronger than the other RDBMS.



\subsection{Heavy load}

\begin{figure}[H]
  \vspace{-10pt}
  \centering
  \centerline{\includegraphics[width=1\textwidth]{no_push/short-oss-impact-high.png}}
  \vspace{-5pt}
  \caption{Impact of the Number of OpenStack Instances on the Completion Time under Heavy Load – Median Time for Each Operations – Authenticate and Validate Keystone Token Scenario.}
  \vspace{-5pt}
  \label{fig:short-oss-impact-high}
\end{figure}



Under a heavy load, once again, we can see on figure \ref{fig:oss-impact-high} and even quicker on figure \ref{fig:short-oss-impact-high}, that MariaDB Galera Cluster has some real difficulties to handle a heavy load under 45 nodes, since 3 scenarios crashed the RDBMS. As explained in the previous section, we do not really know why these failures happen.

The centralized MariaDB has really good results, but CockroachDB mostly handles well the increasing number of nodes, keeping the scaling linear, even though the slope is stronger on the 3 last scenarios.


\section{Delay Impact}

In this test, the size of the database cluster is 9 and the delay varies between LAN, 100 and 300 ms of RTT. The test evaluates how the completion time of Rally scenarios varies, depending of the RTT between nodes of the swarm.



\subsection{Light load}

\begin{figure}[H]
  \vspace{-10pt}
  \centering
  \centerline{\includegraphics[width=1\textwidth]{no_push/short-delay-impact-light.png}}
  \vspace{-5pt}
  \caption{Impact of the Delay on the Completion Time under Light Load – Median Time for Each Operations – Authenticate and Validate Keystone Token Scenario.}
  \vspace{-5pt}
  \label{fig:short-delay-impact-light}
\end{figure}

Figure \ref{fig:delay-impact-light} is the median time for each operations under a light load, depending on the network delay.

MariaDB handles the delay pretty well despite the fact the request have to travel through the network (as we can see on figure \ref{fig:short-delay-impact-light}). On the other hand, CockroachDB has huge difficulties to serve operations under a high latency.


The figure \ref{fig:delay-impact-light-cdf} helps understand what happens; CockroachDB has to satisfy the quorum before making any write. This means it has to pay the cost of latency for every write.

Here, CockroachDB has another problem to cope with the increasing delay: when two concurrent transaction occur, it uses a deadlock and Keystone handles it with an exponential backoff. This means it will pick up a wait time to retry between 0 and 1, 0-2, 0-4, 0-8, etc. But because of the delay, a transaction takes at least 300ms, so any retry smaller than the RTT will only add more contention.

\subsection{Heavy load}

\begin{figure}[H]
  \vspace{-10pt}
  \centering
  \centerline{\includegraphics[width=1\textwidth]{no_push/short-delay-impact-high.png}}
  \vspace{-5pt}
  \caption{Impact of the Delay on the Completion Time under Heavy Load – Median Time for Each Operations – Authenticate and Validate Keystone Token Scenario.}
  \vspace{-5pt}
  \label{fig:short-delay-impact-high}
\end{figure}

Figure \ref{fig:short-delay-impact-high} shows that the tendency is getting worse under a high load, as could be expected. MariaDB Galera Cluster still manages to survive because of the way it only broadcast the changes to apply and then use a deterministic certification to commit, so the latency does not really cost to the system. The centralized MariaDB begins to have some difficulties to manage such a load with a high latency.

Figure \ref{fig:delay-impact-high} shows that CockroachDB even fail to tackle 300ms of RTT for two scenarios. This is due to the fact that after 10 retries, OpenStack Oslo.db discards the transaction. We can already see on the CDF plot from \ref{fig:delay-impact-high-cdf} on the scenarios that fail, under 150 ms of RTT, there is a huge snowball effect: operations take longer and longer for most of them. But we know that CockroachDB uses replications zones so we can optimize the data location to avoid this high delays.

\section{Locality Impact}

This study considers two kinds of OpenStack instances deployments. This first one, called uniform, defines a uniform distribution of the network latency between OpenStack instances. For instance, 300 ms of RTT between all the 9 OpenStack instances (this is seen in the previous section). The second deployment, called hierarchical, maps to a more realistic view, like in cloud computing, with groups of OpenStack instances connected through a low latency network (e.g., 3 OpenStack instances per group deployed in the same country, and accessible within 20 ms of RTT), and high latency network between groups (e.g. 150 ms of RTT between groups deployed in different countries). This deployment has been described in section \ref{subsubsec:replication-zones}.

In all the figures, the completion time is the ordinate as usual, but the bars are respectively the topology with one replicas in each group, two replicas in the first group and one in the second and finally all replicas in the first group.

\begin{figure}[H]
  \vspace{-10pt}
  \centering
  \centerline{\includegraphics[width=0.6\textwidth]{no_push/short-zones-impact-light.png}}
  \vspace{-5pt}
  \caption{Impact of the Locality on the Completion Time under Light Load – Median Time for Each Operations – Create User and Update its Password Scenario.}
  \vspace{-5pt}
  \label{fig:short-zones-impact-light}
\end{figure}

As expected, because the quorum can be achieved with only two replicas out of the three, we can see from one glimpse at figure \ref{fig:short-zones-impact-light} that CockroachDB handles perfectly well under high latency when pulling the data closer to the client since it does not have to pay the cost of the latency.

Figure \ref{fig:zones-impact-light} confirms that, except for the first scenario, CockroachDB is really efficient when considering locality by satisfying the quorum on closer servers. But it was expected, since we ``cheated'' by bypassing the network delay since the leaseholder does not need to wait for the answer from the replicas across the high delay. We can see on the CDF (figure \ref{fig:zones-impact-light-cdf}), that we still have the ``snowball effect'' for the first version (blue line), since there is the 300ms of RTT to contact the other replicas.

We can also see that apart from two scenarios, MariaDB Galera Cluster results are not so good. There is also a strange result on the first scenarios, but we could not really explain this. We can see on the corresponding CDF that at some point, one operation takes a lot of time, then it grows more or less normally, then another one take a lot of time. There was clearly a problem for these operations, but I could not find what.


\addcontentsline{toc}{chapter}{\textbf{Conclusion}}
\chapter*{Conclusion}

An ever growing number of applications requires a wide distribution over the globe of computing resourcing, such as the Internet of Things (IoT). To manage this kind of ``Edge Computing'', the Discovery initiative chose to modify the OpenStack platform in order to avoid reinventing the wheel.

Some OpenStack components do not fulfill edge infrastructure needs. We wanted to see the Relational DataBase Management System cope with lots of connections and an intermittent network access. Furthermore, in order to operate OpenStack in the context of a massively distributed edge infrastucture, we think it is crucial to consider the locality of the client.

To test the performances of different backends for Keystone, OpenStack authentication service, and to better have an idea of the behavior of OpenStack in the context of the edge using locality, we made an evaluation of the performances of different RDBMS, using different topologies.

In order to get this evaluation, I developed Juice, a framework to deploy OpenStack using different backends on a varying number of servers, varying network delays, and a simulation of an edge infrastructure, with heterogenous network delays.

The results of this evaluation show that, though it is a single point of failure, a centralized MariaDB works pretty well under an increasing number of nodes and latency. MariaDB Galera Cluster handles pretty well high latencies despite what we first thought, but has serious difficulties to cope with a high number of nodes. Finally, CockroachdB manages pretty well numerous nodes, but can't handle high delays.

What was even more interesting were the results from the usage of the replication zones from CockroachDB. This showed that we can bypass the difficulties of a distributed database by considering the locality of the client. There are still disadvantages about CockroachDB; e.g. the fact that the replication zones can only be defined for the entire table, and not only on a row level, which would be what we want to only to this for some pieces of data, and not the entire database.

This means that, though CockroachDB is not entirely what we need, it has proved that considering the locality of the client is a good way to continue the work for the Discovery initiative.


\clearpage
\appendix
%\addcontentsline{toc}{section}{\textbf{Appendix}}
\chapter{Appendix}
\label{sec:appendix}



\section{Detailed Rally scenarios}
\label{sec:detail-rally}

(Extract from the blog article, from the work of Ronan Cherrueau)

The \%reads/\%writes ratio is computed on Mariadb. The gauging code reads values of status variables \verb+Com\_xxx+ that provide statement counts over all connections (with \verb+xxx+ standing for \verb+SELECT+, \verb+DELETE+, \verb+INSERT+, \verb+UPDATE+, \verb+REPLACE+ statements). The SQL query that does this job is available in listing \ref{lst:rw} and returns the total number of reads and writes since the database started. That SQL query is called before and after the execution of one Rally scenario. After and before values are then subtracted to compute the number of reads and writes performed during the scenario and finally, compared to compute the ratio.

\begin{lstlisting}[caption={Total number of reads and writes performed on MariaDB since the last reboot}\label{lst:rw},language=SQL]
SELECT
  SUM(IF(variable_name = 'Com_select', variable_value, 0))
     AS `Total reads`,
  SUM(IF(variable_name IN ('Com_delete',
                           'Com_insert',
                           'Com_update',
                           'Com_replace'), variable_value, 0))
     AS `Total writes`
FROM  information_schema.GLOBAL_STATUS;
\end{lstlisting}

Note that \%reads/\%writes may be a little bit more in favor of reads than what it is presented here because the following also takes into account the creation/deletion of the Rally context. A basic Rally context for a Keystone scenario is \verb+{"admin_cleanup@openstack": ["keystone"]}+. We are not sure what does this context do exactly though, maybe it only creates an admin user… This context may be extended by other inserts specified in the scenario definition (under the \verb+context+ key; see scenario definition for keystone/create-add-and-list-user-roles).

The Juice implementation for this gauging is available on GitHub at \url{https://github.com/rcherrueau/juice/blob/02af922a7c3221462d7106dfb2751b3be709a4d5/experiments/read-write-ratio.py}.


\subsection{keystone/authenticate-user-and-validate-token}

Description: authenticate and validate a keystone token.

Definition Code: \url{https://github.com/openstack/rally-openstack/blob/master/samples/tasks/scenarios/keystone/authenticate-user-and-validate-token.yaml}

Source Code: \url{https://github.com/openstack/rally-openstack/blob/master/rally\_openstack/scenarios/keystone/basic.py#L111-L120}

List of keystone functionalities:
\begin{itemize}
\item    keystone\_v3.fetch\_token
\item    keystone\_v3.validate\_token
\end{itemize}
\%Reads/\%Writes: 96.46/3.54

Number of runs: 20

\subsection{keystone/create-add-and-list-user-roles}

Description: create user role, add it and list user roles for given user.

Definition Code: \url{https://github.com/openstack/rally-openstack/blob/master/samples/tasks/scenarios/keystone/create-add-and-list-user-roles.yaml}

Source Code: \url{https://github.com/openstack/rally-openstack/blob/master/rally\_openstack/scenarios/keystone/basic.py#L214-L228}

List of keystone functionalities:
\begin{itemize}
\item    keystone\_v3.create\_role
\item    keystone\_v3.add\_role
\item    keystone\_v3.list\_roles
\end{itemize}
\%Reads/\%Writes: 96.22/3.78

Number of runs: 100

\subsection{keystone/create-and-list-tenants}

Description: create a keystone tenant with random name and list all tenants.

Definition Code: \url{https://github.com/openstack/rally-openstack/blob/master/samples/tasks/scenarios/keystone/create-and-list-tenants.yaml}

Source Code: \url{https://github.com/openstack/rally-openstack/blob/master/rally\_openstack/scenarios/keystone/basic.py#L166-L181}

List of keystone functionalities
\begin{itemize}
\item    keystone\_v3.create\_project
\item    keystone\_v3.list\_projects
\end{itemize}
\%Reads/\%Writes: 92.12/7.88

Number of runs: 10

\subsection{keystone/get-entities}

Description: get instance of a tenant, user, role and service by id’s. An ephemeral tenant, user, and role are each created. By default, fetches the ’keystone’ service.

List of keystone functionalities:
\begin{itemize}
\item    keystone\_v3.create\_project
\item    keystone\_v3.create\_user
\item    keystone\_v3.create\_role
    \begin{itemize}
    \item    keystone\_v3.list\_roles
    \item    keystone\_v3.add\_role
    \end{itemize}
\item    keystone\_v3.get\_project
\item    keystone\_v3.get\_user
\item    keystone\_v3.get\_role
\item    keystone\_v3.list\_services
\item    keystone\_v3.get\_services
\end{itemize}
\%Reads/\%Writes: 91.9/8.1

Definition Code: \url{https://github.com/openstack/rally-openstack/blob/master/samples/tasks/scenarios/keystone/get-entities.yaml}

Source Code: \url{https://github.com/openstack/rally-openstack/blob/master/rally\_openstack/scenarios/keystone/basic.py#L231-L261}

Number of runs: 100


\subsection{keystone/create-user-update-password}

Description: create user and update password for that user.

List of keystone functionalities:
\begin{itemize}
\item    keystone\_v3.create\_user
\item    keystone\_v3.update\_user
\end{itemize}
\%Reads/\%Writes: 89.79/10.21

Definition Code: \url{https://github.com/openstack/rally-openstack/blob/master/samples/tasks/scenarios/keystone/create-user-update-password.yaml}

Source Code: \url{https://github.com/openstack/rally-openstack/blob/master/rally\_openstack/scenarios/keystone/basic.py#L306-L320}

Number of runs: 100


\subsection{keystone/create-user-set-enabled-and-delete}

Description: create a keystone user, enable or disable it, and delete it.

List of keystone functionalities:
\begin{itemize}
\item    keystone\_v3.create\_user
\item    keystone\_v3.update\_user
\item    keystone\_v3.delete\_user
\end{itemize}
\%Reads/\%Writes: 91.07/8.93

Definition Code: \url{https://github.com/openstack/rally-openstack/blob/master/samples/tasks/scenarios/keystone/create-user-set-enabled-and-delete.yaml}

Source Code: \url{https://github.com/openstack/rally-openstack/blob/master/rally\_openstack/scenarios/keystone/basic.py#L75-L91}

Number of runs: 100

\subsection{keystone/create-and-list-users}

Description: create a keystone user with random name and list all users.

List of keystone functionalities:
\begin{itemize}
\item    keystone\_v3.create\_user
\item    keystone\_v3.list\_users
\end{itemize}
\%Reads/\%Writes: 92.05/7.95

Definition Code: \url{https://github.com/openstack/rally-openstack/blob/master/samples/tasks/scenarios/keystone/create-and-list-users.yaml}

Source Code: \url{https://github.com/openstack/rally-openstack/blob/master/rally\_openstack/scenarios/keystone/basic.py#L145-L163}

Number of runs: 100


\clearpage
\phantomsection
\addcontentsline{toc}{chapter}{\textbf{Bibliography}}


\printbibliography
%% \printbibliography[title={Book references},type=book]
%% \printbibliography[title={Article references},type=article]
%% \printbibliography[title={Other references}, nottype=article, nottype=book]



\end{document}
